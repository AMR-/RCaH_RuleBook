\newpage
\section{Robo-Nurse}

The robot is assisting an elderly person with getting her pills and responding to observed activities.

\subsection{Focus}

This test focuses mainly on Human-Robot Interaction and Activity Recognition.

\subsection{Task}
\begin{enumerate}
	\item \textbf{Start}: The robot is in a corner of the living room, the patient is sitting in the same room. 

	\item \textbf{Move to the patient}: The patient (lets call her Granny) calls for robot assistance using her voice or by waving arms.

	\item \textbf{Asking for pills}: Granny asks the robot for her pills which are in bottles located on a shelf nearby. 
	  This is the start command for the robot and may be defined by the team. 
	  Because this test is an integrated scenario, the command should make sense in the context fo the test.
	  For example ``Continue'' does not make sense but ``I need my pills, robot'' and ``Robot, please get me my pills'' do make sense.
	  The team must instruct ``granny'' about what to say.

	\item \textbf{Describe and choose pills}: On the shelf, there are multiple bottles with pills and the robot must asks Granny which bottle she needs.
	\begin{itemize}
		\item The robot must indicate what bottles are on the shelf by briefly describing each bottle.
		\textbf{the faster the robot starts describing (i.e.~finished recognizing) after arrival (standing still in front of the shelf) at the shelf, the better:} faster recognition gives more points. 
		  For scoring, the time between the robot's arrival and the start of speaking the description is taken.
		\begin{itemize}
 			\item \quotes{The leftmost one}
  			\item \quotes{The \textit{color} bottle}
  			\item \quotes{The big/small bottle}
  			\item Any other description the robot understands and spoke out loud to Granny. E.g.~if a robot can do text recognition and read each label to Granny, she may reply with e.g.~\quotes{Aspirin}.
 		\end{itemize}
 		\item The bottles will be previously unseen by the robots and thus cannot be trained. 
 		The descriptions thus have to really be created from observations, on the fly.
 	\end{itemize}

 	\item \textbf{Grasp \& handover pills}: The robot must grasp the indicated bottle of pills and hand it over to Granny. The handover to Granny must be \quotes{natural}, without a voice confirmation of when to let the pills go etc.~Granny will take the pills from the robot's hand and the robot must open its hand.

 	\item \textbf{Open pills bottle [Optional]:} After grasping a pills bottle, the robot may try to open it before delivering it to Granny.

 	\item \textbf{Give a single pill [Optional]:} If the robot succeeds opening the pills bottle, it may try to fetch a single pill from it to delivery to Granny.

 	\item \textbf{Activity Recognition}: One of the activities below happens and the robot must act accordingly:
 	\begin{itemize}
 		\item \textbf{Drop blanket}: Granny's stands up and sits down immediately. Her blanket falls from her lap to the ground. The robot must \textbf{pick up the blanket} and hand it to Granny.

 		\item \textbf{Fall}: Granny stands up from her chair and falls. The robot must \textbf{hand Granny a phone}. The phone will be laying nearby, e.g.~on the coffee table in the living room. \\
		\textbf{[Optional]:} Robot may use Smart House option to do a phone call instead of delivering the phone to Granny.

 		\item \textbf{Walk and sit}: Granny walks to a table with her walking stick/cane. Robot must follow her and take the walking stick from Granny after she sits down on a nearby chair.
 	\end{itemize}
\end{enumerate}


\subsection{Additional rules and remarks}
\begin{enumerate}
	\item \textbf{Continue Rule:} The CONTINUE rule may be applied several times in the Conversation part of the test (Section \refsec{rule:asrcontinue}).

	\item \textbf{Make it fast:} Description of objects should be fast, as is reflected in the scoring.
	% \item The bottles of pills are not known beforehand and must be recognized and described on the spot by the robot. 
	
	\item \textbf{Opening pill bottles:} Provided pills' bottles will be chosen so they can be opened easily by twist, i.e.~no push and twist, no uncap, no seals nor any other complex opening method.

	\item \textbf{Optional tasks:} The test includes optional tasks (such as describe unknown objects, opening bottles, and grasping very small objects) which are not required to be performed as part of the overall test but brings an additional scoring for solving it. Team leader must contact a TC member to request optional tasks to be available.

	\item \textbf{Smart-house:} The arena-house may have enabled official smart-house devices (Section \refsec{rule:smarthomedevices}), there are additional scoring for interacting with the house.
\end{enumerate}

\subsection{Referee instructions}

The referee needs to
\begin{itemize}
	\item Place the pill bottles on the shelf.
	\item Place the phone on announced position.
\end{itemize}

\subsection{OC instructions}

\textbf{2 hours before the test}
\begin{itemize}
	\item Announce the room where the patient is.
	\item Find at least 5 items that can serve as pill bottles.
	\item Announce the room where the phone is.
\end{itemize}

\textbf{During the test}
\begin{itemize}
	\item Instruct Granny which pills she wants.
	\item Instruct Granny which of the 3 actions to perform.
\end{itemize}

\newpage 
\subsection{Score sheet}
The maximum time for this test is 10 minutes.

\small\begin{scorelist}

	\scoreheading{Attending request}
	\scoreitem{20}{Reach patient after being called}
	\scoreitem{10}{Await command to get pills}

	\scoreheading{Describing pills}
	\scoreitem{50}{Real time description (given upon arrival)}
	\scoreitem{30}{Description given within $t \leq 5$ seconds}
	\scoreitem{20}{Description given within $5 < t \leq 15$ seconds}
	\scoreitem{10}{Description given within $15 < t \leq 30$ seconds}
	\scoreitem{00}{Description given within $ t \geq 30$ seconds}

	\scoreheading{Picking pills}
	\scoreitem{40}{Choose the correct pills}
	\scoreitem{20}{Grasp the correct pills}
	\scoreitem{5}{Grasp wrong pills}

	\scoreheading{Pills handover}
	\scoreitem{20}{Natural delivery (no instructions are given to operator)}
	\scoreitem{10}{Assisted delivery (operator instructs robot for delivery)}

	\scoreheading{Activity recognition}
	\scoreitem{50}{Granny trying to reach drop blanket}
	\scoreitem{5}{Falling Granny}
	\scoreitem{5}{Granny stands up and walk away + sit}

	\scoreheading{Response to activity}
	\scoreitem{40}{Pickup the blanket + give the blanket}
	\scoreitem{40}{Grasp phone + give phone}
	\scoreitem{40}{Take walking stick / cane}

	\scoreheading{Bonuses (up to 180 points)}
	\scoreitem{10}{Using Smart House to call instead of grasping \& giving the phone}
	\scoreitem{30}{Describe unknown pills' bottles}
	\scoreitem{40}{Opening the pill bottle (with a screw cap)}
	\scoreitem{100}{Picking a single pill from the bottle}
	
	\setTotalScore{250}
\end{scorelist}



% Local Variables:
% TeX-master: "Rulebook"
% End:


% Local Variables:
% TeX-master: "Rulebook"
% End:
