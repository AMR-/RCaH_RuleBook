\newcommand{\bonusRobotCoop}{5~}

\section{Open Challenge}

During the Open Challenge teams are encouraged to demonstrate recent research results and the best of the robots' abilities. It focuses on the demonstration of new approaches/applications, human-robot interaction and scientific value. To participate in this test it is required to participate in any other test from Stage~II.

\subsection{Task}

The Open Challenge is an open demonstration which means that the teams may demonstrate anything they like with an high difficulty degree. The performance of the teams is evaluated by a jury consisting of all members of the technical committee.

The procedure for the challenge and the timing of slots is as follows:
\begin{enumerate}
  {\bf\item Setup and demonstration:} The team has a maximum of seven minutes for setup and demonstration. During the demonstration, the robot must perform at least 1 complex task (see \refsec{chap:example-skills}for a list of examples on each category), preferably including Human-Robot interaction or a Smart House environment, to be evaluated by the Technical Committee. During Setup Time and before the demonstration begins, the team leader is allowed to \emph{very} briefly the address the problem and the demonstrated approach (maximum time is one minute).

	\begin{enumerate}
		\item If available, video projector or screen may be used to visualize robot's internals, e.g., perceptions.
		\item It is important to note that the jury may decide to end the demonstration if there is nothing happening or nothing new is happening.
	\end{enumerate}
  {\bf\item Interview and cleanup:} After the demonstration, there is another three minutes where the team answers questions by the jury members.

  During the interview time, the team has to undo its changes to the environment.
\end{enumerate}

\subsection{Changes to the environment}
\begin{enumerate}
  \item Making changes: As in the other open demonstrations, teams are allowed to make modifications to the arena as they like, but under the condition that they are reversible.
  \item Undoing changes: In the interview and cleanup team, changes need to be made undone by the team. The team has to leave the arena in the very same condition they entered it.
\end{enumerate}

\subsection{Jury evaluation}
The jury is constituted of members of the technical committee. Evaluation is based on the following criteria:

\begin{enumerate}
	% \item Novelty (Seen before in @Home?)
	\item Overall demonstration
	\item Scientific contribution (Is that new in @Home?)
	\item Robot autonomy in the demonstration
	\item Difficulty of the performance (How difficult is it?)
	\item Success of the performance (The robot did it?)
	\item Contribution for @Home (can other teams use the solution?)
\end{enumerate}

\subsection{Additional rules and remarks}
\begin{enumerate}
	\item \textbf{Start signal:} There is no standard start-signal for this test.
	\item \textbf{Abort on request:} At any time during the demonstration, the jury may interrupt and abort the demonstration:
	\begin{enumerate}
		\item if nothing is shown: in case of longer delays (more than one minute), e.g., when the robot does not start or when it got stuck;
		\item if nothing new is shown: the demonstrated abilities were already shown in previous tests (to avoid dull demonstrations and push teams to present novel ideas).
	\end{enumerate}

	\item \textbf{Team-team-interaction:}  An extra bonus of up to \bonusRobotCoop points can be earned if robots from two teams (4 robots maximum, 2 from each team) successfully collaborate (robot-robot interaction).
	\begin{enumerate}
		\item This bonus is earned for both teams.
		\item The robot(s) of the other team must only play a minor role in the total demonstration.
		\item It must be made clear that the demonstrations from the two teams are not similar, otherwise the points cannot be awarded.
		\item In case a team receives two (or more) bonuses, the maximum bonus will be taken.
		\item The collaboration is possible even if one of the two teams has not reached Stage 2.
		\item The team which does not participate in Stage 2 receives no points for this test.
	\end{enumerate}
\end{enumerate}


% Local Variables:
% TeX-master: "Rulebook"
% End:
