\subsection{Person recognition test}

The robot has to identify the operator within a crowd and state information about the operator and the crowd.

\subsubsection{Focus}

This test focuses on people detection and recognition; as well as pose recognition and human-robot interaction with unknown people.

\subsubsection{Setup}

\begin{enumerate}
\item \textbf{Operator:} A \quotes{professional} operator is selected by the TC to test the robot.
\item \textbf{Other people} There are no restrictions on other people walking by or standing around throughout the complete task.
\end{enumerate}

\subsubsection{Task}
\paragraph{This test may also be held outside the arena}

\begin{enumerate}

\item \textbf{Start:} The robot starts at a designated starting position, and waits for the \quotes{professional} operator. When the referees start the time, the team is not allowed to instruct the operator.
\item \textbf{Memorizing the operator:} The robot has to memorize the operator. During this phase, the robot may instruct the operator to follow a certain setup procedure.
\item \textbf{Wait 1 minute:} Once the robot states it has finished memorizing the operator, it must stand still for 1 minute. The operator will walk around the robot and locate behind him within a crowd.
\item \textbf{Find the crowd:} After the time elapses, the robot must turn about 180°, approach to the crowd and start looking for the operator.
\begin{itemize}
\item \textbf{Crowd size:} The crowd may contain between 5 and 10 people, standing or sitting or lying within an  area of 5 meters (diameter).
\item \textbf{Crowd position:} The crowd will be located behind the robot at a distance between 2 and 3 meters apart.
\end{itemize}
\item \textbf{Find the operator:} Once the crowd has been located, the robot must greet the operator and state the gender, mood (happy, sad, angry, etc.), pose (sitting, standing, rising arms, etc.). Also, it must give a description of the operator; as well as pointing or approaching to it.

\textit{I found you operator. You are the blond smiling girl with blue eyes and red skirt sitting in the middle of the crowd.}

\item \textbf{Describe the crowd}: Finally, robot must tell the size of the crowd and how many men, women and even children are.
\end{enumerate}

\subsubsection{Additional rules and remarks}
\begin{enumerate}
\item \textbf{Preparation:} The robot needs to wait for at least 1 min before the operator appears in front of the robot. During this waiting time the team is not allowed to touch the robot.
\item \textbf{Disturbances from outside:} If a person from the audience (severely) interferes with the robot in a way that makes it impossible to solve the task, the team may repeat the test immediately.
\item \textbf{Instruction:} The robot interacts with the operator, not the team. That is, the team is not allowed to instruct the operator.
\end{enumerate}


%\subsubsection{Referee instructions}
%
%The referee needs to
%\begin{itemize}
%\item 
%\item 
%\end{itemize}

\subsubsection{OC instructions}

\textbf{2 hours before the test}
\begin{itemize}
\item Select the \quotes{professional} operator(s).
\item Select the crowd.
\end{itemize}

\textbf{During the test}
\begin{itemize}
\item Check save operation of the robot; the robot needs to be stopped immediately if a person is going to be touched by the robot
\end{itemize}

\subsubsection{Score sheet}
The maximum time for this test is 5 minutes.

\begin{tabularx}{\textwidth}{ X r }
	\textbf{Action} & \textbf{Score} \\ \hline
	\textbi{Operator}  \\
	Approach or point at the operator & 1.0 \\
	Correctly state operator's gender & 1.0 \\
	Correctly state operator's appearance (eyes, hair, etc.) & 1.0 \\
	Correctly state operator's dressing & 1.0 \\
	Correctly state operator's pose & 3.0 \\
	Correctly state operator's mood & 3.0 \\
	\\
	\textbi{Crowd} \\
	Correctly state crowd's size & 1.0 \\
	Correctly state crowd's number of men & 2.0 \\
	Correctly state crowd's number of women & 2.0 \\
	\\ \hline
	\textbf{Total score} (excluding penalties and bonuses) & 15.0 \\
\end{tabularx}