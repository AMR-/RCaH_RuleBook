\section{General Purpose Service Robot}

This test evaluates Human-Robot Interaction and the integration of the abilities of the robot tested in stage I. In this test the robot has to solve multiple tasks upon request. That is, the test is not incorporated into a (predefined) story and there is neither a predefined order of tasks nor a predefined set of actions. The actions that are to be carried out by the robot are randomly generated by the referees and are composed by 3 subtasks which include navigation, human-robot interaction and robot-object interaction.

The command is composed by three actions, which the robot has to show it has recognized. The robot may repeat the understood command and ask for confirmation. If it can't recognize the command correctly, it can also ask the speaker to repeat the complete command, or ask for further information.

\subsection{Focus}
This test particularly focuses on the following aspects:
\begin{itemize}
	\item No predefined order of actions to carry out (to get away from state machine-like behavior programming).
	\item Increased complexity in speech recognition (possible commands are less restricted in both actions/operators and arguments/objects, commands can include multiple objects, e.g., \quotes{put the apple on the kitchen table})
\end{itemize}

\subsection{Task}

\begin{enumerate}
	\item \textbf{Entering and command retrieval:} The robot enters the arena and drives to a designated position where it has to wait for further commands.
	\item \textbf{Command generation:} A command is generated randomly, depending on the command category chosen by the team (see below). \\

	\item \textbf{Command categories:} All possible actions has been classified previously by the TC according to their difficulty level. The team may choose from the following three categories:
	\begin{enumerate}
		\item \textbf{Category I:} Tasks with a low degree of difficulty (easy to solve). This category includes indoor navigation, grasping known objects, answering questions (from the predefined set of questions), etc.
		\item \textbf{Category II:} Tasks with a moderate degree of difficulty. This category includes following a human, indoor navigation in crowded environments, grasping alike objects, find a calling person (waving or shouting), etc.
		\item \textbf{Category III:} Tasks with a high degree of difficulty (challenging). This category includes counting objects in a shelf or people in a crowd, deal with incomplete information (human-robot interaction), state the gender or gesture or a person, handling clothes, etc.
	\end{enumerate}

	\item \textbf{Task assignment:} The robot is given the command by the operator and may directly start to work on the task assignment.
	\item \textbf{Exiting the arena:} After accomplishing the assigned task, the robot has to leave the arena.
\end{enumerate}

\subsection{Commands and actions}
The command is composed by three actions, which the robot has to show it has recognized. The robot may repeat the understood command and ask for confirmation. If it can't recognize the command correctly, it can also ask the speaker to repeat the complete command. If the robot fails to understand the given commands, it may ask to the operator to repeat them up to three times, if it fails the team may opt to use the Continue rule (Section \refsec{rule:asrcontinue}). In case the robot has understood partially the command, it may ask the operator for additional information (e.g.~\quotes{did you say apple juice or pineapple juice?}).

Required in this test are:
\begin{enumerate}
	\item abilities from stage I forming a set of actions $A$ (e.g., following a person, finding a random person, finding a person after memorizing her; finding, recognizing, grasping, and delivering objects, etc.),
	\item a set of people $P$,
	\item a set of questions $Q$,
	\item a set of objects $O$ (the same set as used as in the other tests),
	\item a set of locations $L$ (the same set as used as in the other tests).
\end{enumerate}

Each task assignment contains an action $a \in A$ and, depending on the respective action an object $o \in O$, a location $l \in L$, a question $q \in Q$, a person $p \in P$ to interact with, or a combination of those. The set of actions is not given beforehand, instead, teams should identify the abilities from Stage I by themselves (and find synonyms for that). That is, $L$, $O$ and $Q$ are known in advance (provided during setup days), but $A$ has to be \quotes{found out} by the teams (e.g.~taken from freely available ontologies, synonym searches etc.). For the actions $A$ are going to be used common synonyms (like \quotes{go to}, \quotes{move to}, \quotes{drive to}, and \quotes{navigate to} to describe navigation). For the people $P$, any person willing to operate the robot in a natural way can be expected, however, \quotes{Professional Operators} are more likely to be used.

\paragraph{Command examples}
\begin{itemize}
	\item Go to the bedroom, find a person and tell the time.
	\item Go to the kitchen, find a person and follow her.
	\item Go to the dinner-table, grasp the crackers, and take them to the side-table.
	\item Go to the shelf, count the drinks and report to me.
	\item Take this object and bring it to Susan at the hall.
	\item Bring a coke to the person in the living room and answer him a question.
	\item Offer a drink to the person at the door (robot needs to solve which drink will be delivered).
\end{itemize}

\subsection{Additional rules and remarks}
\begin{enumerate}
	\item \textbf{Referees:} Since the score system in this test involves a subjective evaluation of the robot's behavior, the referees are EC/TC members.
	\item \textbf{Operator:}
	\begin{itemize}
		\item The person operating the robot is one of the referees (default operator).
		\item If the robot appears to consistently not be able to understand the operator, the referees ask the team to continue with a custom operator (Section \refsec{rule:operator}).
		\item With the custom operator, the team can only score 50\% of the points for the respective command.
	\end{itemize}
	\item \textbf{Following people} 
	\begin{enumerate}
		\item \textbf{Instruction:} The robot interacts with the operator, \emph{not} the team. That is, the team is not allowed to briefly instruct the operator.
		\item \textbf{Natural walking:} The operator has to walk \quotes{naturally}, i.e., move forward facing forward. The operator is not allowed to walk back, stand still, signal the robot or follow some re-calibration procedure.
		\item \textbf{Asking for passage:} The robot is allowed to (gently) ask people to step aside.
		\item \textbf{Stopping:} The robot must decide when to stop following a person, either because it was instructed to follow her to a certain location, because it was asked to stop by the operator or because the test time is running out. In any case, the robot should state the reason why it changes its behavior.
	\end{enumerate}
\end{enumerate}

\subsection{Referee and OC instructions}
\textbf{2h before test:}
\begin{itemize}
\item Specify and announce the entrance and exit door
\end{itemize}
\textbf{During the test:}
\begin{itemize}
\item Generate random sentences by an automatic sentence generator
\end{itemize}

\newpage
\subsection{Score sheet}
The maximum time for this test is 8 minutes.

\begin{tabularx}{\textwidth}{ X r }

	\textbf{Action} & \textbf{Score} \\ \hline
	\textbi{Getting instructions}  \\
	Understanding the set of actions on the first  attempt & 4.0 \\
	Understanding the set of actions on the second attempt & 2.0 \\
	Understanding the set of actions on the third  attempt & 1.0 \\
	Reduction of points for every command provided by a team member & $0.5 \times -1$ \\
	\\
	\textbi{Performing the task: Category I}  \\
	Performing the first task correctly & 1.0 \\
	Performing the first and second task correctly & 1.0 \\
	Successfully solving the complete command & 3.0 \\
	\\
	\textbi{Performing the task: Category II}  \\
	Performing the first task correctly & 2.0 \\
	Performing the first and second task correctly & 3.0 \\
	Successfully solving the complete command & 5.0 \\
	\\
	\textbi{Performing the task: Category III}  \\
	Asking reasonable questions to obtain missing information & 2.0 \\
	Performing the first task correctly & 4.0 \\
	Performing the first and second task correctly & 6.0 \\
	Successfully solving the complete command & 8.0 \\
	\\
	\textbi{Leave the arena}  \\
	Exiting the arena & 1.0 \\
	
	\\ \hline
	\textbf{Total score} (excluding penalties and bonuses) & 25.0 \\
\end{tabularx}


% Local Variables:
% TeX-master: "Rulebook"
% End:
