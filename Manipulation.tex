\section{Manipulation and Object recognition}

The robot must reach a bookcase or table in which there are several objects at different heights (top-most or bottom-most shelf) or positions (one very close from other at the table). The robot must then identify and grasp all of those objects and put them all together into a new, easy-to-reach location (middle shelf or another table).

\subsection{Goal}
The robot has to identify, grasp and correctly place several objects at different heights or positions.


\subsection{Focus}

This test focuses on object detection, and manipulation; as well as object recognition.

\subsection{Setup}
\paragraph{This test may also be held outside the arena}

\begin{enumerate}
\item \textbf{Location:} One of the bookcases of the apartment is used for this test. The robot will start at a random distance between 1.0m and 1.5m from the bookcase.
\footnote{The minimum and maximum distance of the objects in the shelf is still being discussed. This value may change.}
The bookcase has at least 5 shelves between 0.30m and 1.80m from the ground. One of the shelves in the middle is empty.
\item \textbf{Objects:} The bookcase contains 5 objects from the set of predefined objects.
\item \textbf{Object distribution:} The objects are located as follows:
\begin{enumerate}
\item An easy-to-grasp (bottle, small cereal box, can, etc...) object on a top shelf.
\item An easy-to-grasp (bottle, small cereal box, can, etc...) object on a bottom shelf.
\item A heavy object (1L milk box, big soda bottle, etc...) on a middle shelf.
\item A hard-to-grasp object (cloth, apple, banana, etc...) on a middle shelf.
\item A hard-to-reach object (near a corner or a wall) on a middle shelf.
\end{enumerate}
Please note that may be more than one object in each shelf.
\end{enumerate}

\subsection{Task}
\begin{enumerate}
\item \textbf{Searching for objects:} When told so by an operator, the robot approaches to the shelf and start searching for objects.
\item \textbf{Grasping objects:} Any object found by the robot may be grasped by it. Before or right after grasping the object, the robot has to announce which object it has found.
\item \textbf{Placing objects:} After grasping the object, the robot has to safely place it (Section \ref{rule:scenario_objects}) on the empty shelf a the middle of the bookcase. The object must stay there for at least 10sec.
\item \textbf{Handling objects multiple times:} Scores can only be gained a single time for each specific object.
\end{enumerate}

\subsection{Additional rules and remarks}
\begin{enumerate}
\item \textbf{No setup:} The robot must be ready to start the test with a voice command or start button when requested by the referee. There is no setup time.
\item \textbf{Startup:} The robot must be started with a single voice command or via a start button (Section \ref{rule:start_signal}). If the robot is unable to start it must be removed immediately.
\item \textbf{Single try:} The robot must be able to start from the first attempt. There is no restart for this test. If the robot is unable to start it must be removed immediately.
\item \textbf{Collisions:} Slightly touching the shelves or the bookcase is tolerated. Driving over the objects or any other form of a major collision is not allowed, and the referees directly stop the robot (Section \ref{rule:safetyfirst}).
\item \textbf{Object types:} The objects selected from the \textit{Standard Objects Set} will be chosen to be easily detectable and contrasting with the shelf (ex. red or black objects on a white shelf).
\item \textbf{Recognition report:} After the test is completed or the time has run out, the robot may upload a single PDF report file including the list of recognized objects with a picture showing the object, the object name, and the bounding box of the object.
%\item \textbf{QR Codes:} The team may request to use a special set ob objects identified with QR codes if the robot is not able to correctly recognize the objects. The use of this special QR-object-set must be announced to the TC at least on hour before the test starts. When QR Codes are used, no points are given for object recognition.
\end{enumerate}

\subsection{Referee instructions}

The referee needs to
\begin{itemize}
\item Place the objects in the bookcase
\item Make sure there is one empty shelf in the middle of the bookcase
\end{itemize}

\subsection{Score sheet}

The maximum time for this test is 3 minutes.

\begin{tabularx}{\textwidth}{ X r }
	\textbf{Action} & \textbf{Score} \\ \hline
	\textbi{Grasping objects} & \\
	Grasping any object (and successfully lifting it up to at least 5 cm for more than 10 second) & $5 \times 1.0$\\
	\\
	\textbi{Placing objects} & \\
	Placing any object (safely and the objects stands still for more than 10 second) & $5 \times 1.0$\\
	\\
	\textbi{Recognizing objects} & \\
	Every correctly recognized object in the report file & $5 \times 1.0$\\
	\\ \hline
	\textbf{Total score} (excluding penalties and bonuses) & \textbf{15.0}
\end{tabularx}
<<<<<<< HEAD

\footnotetext{The minimum and maximum distance of the objects in the shelf is still being discussed. This value may change. }
=======
>>>>>>> 09b31f9fea75148cf1207f8d4d60144836650682
