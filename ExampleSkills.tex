\chapter{Example Skills}
\label{chap:example-skills}

The following section presents a list of \iterm{Example Skills} with an high degree of difficulty which can be exploited during the \textit{Open Demonstrations} (See \refsec{sec:open-demonstrations}.
Other skills not on this list (yet) may be added as well. If you want to do so, please let the TC know via email (tc@robocupathome.org) for their inclusion on the RuleBook so all teams may also show this skill.

Please note that these examples are to illustrate the level of complexity and applicability that should be shown. For instance, \quotes{Handle a pan} is listed in the category of \textit{Complex manipulation}, but it is extensive to handling pans, pots, woks and any other cookware with handles.

\section{Skills by category}

\subsection{Complex manipulation}
\begin{itemize}
	\item Cook a meal.
	\item Manipulating panels/switches/knobs.
	\item Use/open a fridge/stove/blender/microwave/washing machine.
	\item Iron clothes.
	\item Move a movable object (pole, chair, table).
	\item Pouring liquids/powders.
	\item Operate a water tap.
	\item Handle a pan.
\end{itemize}

\subsection{Complex vision}
\begin{itemize}
	\item Read text from a newspaper.
	\item Handle glass/shiny-metallic objects.
	\item Recognize moods, activities, age, gender.
%	\item Recognize clothes, dressing-styles, fashionable people.
	\item Label unknown objects.
\end{itemize}

\subsection{Complex navigation}
\begin{itemize}
	\item Navigate in (very) crowded environments.
	\item Navigate difficult terrain.
	\item Climb stairs.
	\item Push a wheelchair.
\end{itemize}

\subsection{Robot-Human Interaction}
\begin{itemize}
	\item Collaborative robot-human manipulation.
	\item Maintaining a conversation.
	\item Learning actions on-the-fly.
	\item Learning objects from humans e.g. \quotes{This object is a ...} with an open vocabulary.
	\item Following a human by grasping its hand.
	\item Explain the robot abstract concepts (why people love sunny days).
	\item Arrange unknown random people for a nice photo (no occlusions).
	% \item ask the robot for the answer to the universe, meaning of life and everything else
\end{itemize}

\subsection{Complex action planning} 
\begin{itemize}
	\item Separate clothes for laundry (e.g.~by color)
	\item Arrange a dish-washer.
	\item Take a cup from the cupboard whose location has changed, is closed, or the path to it is blocked (e.g.~by a chair).
	\item Light the way out with a lamp during a general power off.
	\item Arrange unknown random people for a nice photo (no occlusions).
	\item 
\end{itemize}

\subsection{Mapping}
\begin{itemize}
	\item Learn/create a (3D) map on the fly.
	\item Semantically annotate a map on the fly
	\item The robot enters a completely changed arena (furniture moved or even changed), 
	   explores it and is told to go to e.g. a table that is moved or added.
\end{itemize}


% Local Variables:
% TeX-master: "Rulebook"
% End:
