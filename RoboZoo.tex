\section{RoboZoo}
The robots of all teams are presented and arranged in a way that all of them form a zoo-type corridor through which the general audience will walk. Each robot is enclosed within a space that it cannot get out of, and it must perform a show for up to one hour, such as dancing or carrying out any menial task. Each member of the audience who enters the corridor will receive 5 tokens which will be given to his/her top 5 favorite robots. The robot who earns the most tokens wins the contest and gets the maximum score. Points are awarded to the other robots based on the amount of tokens they gathered, proportional to the amount of tokens gotten by the robot that won the contest.

Interaction with the audience is desirable but not mandatory.

\subsection{Enclosed Space Dimension}
The enclosed space is estimated to be around $2 \times 2$ meters. However, teams should expect reasonable deviations in these dimensions, since space in the venue may require smaller enclosed spaces.

\subsection{Security Concerns}
Security is first priority in this competition. To this effect, one team member is required to be inside the enclosed space to ensure that the robot is performing securely. Physical interaction between audience members and the robot is not allowed (i.e. robot handing things to people or shaking hands). Interactions such as talking to the robot, or carrying out face recognition are allowed. To not limit the creativity of the teams in their demonstrations, the robot may hand-out items to the public via using the one team member inside the enclosed space as a type of proxy. In addition, persons from the general public are not allowed inside the enclosed space at any moment.

\paragraph*{Important Note:} Even if people is not allowed to enter the robots' cages, it may happen people (small children) get into the cages. In those cases, robot must be 

\subsection{Restart and Charging}
If the robot requires a restart, the one team member inside the enclosed space may tend to it and restart it as much times as required. However, it is important to note that this test is essentially scored by the general public, and it is reasonable to expect that the audience will not be attracted to a robot being constantly fixed. In addition, since this test may last up to one hour, the robot may require a change of batteries or to use a charging station, which is allowed. However, as pointed out before, this may not be attractive to the audience, so it is recommended to reduce the charging necessities to a minimum. 

\subsection{Additional rules and remarks}

\begin{itemize}
\item \textbf{Gifts:} Robots and team member are \emph{not} allowed to hand out gifts as part of the RoboZoo challenge. 
\item \textbf{Protagonist robots:} Robots must be able to perform autonomously during the test. Team members are \emph{not} allowed to interact with the audience, teach instructions, take part of the show, etc.
\item \textbf{People in the cage:} At any time, one team member must be inside the cage to take care of the robot. More than one team member inside the cage is \emph{not} allowed.
\end{itemize}

\subsection{OC instructions}

2h before test:
\begin{itemize}
\item{Announce to teams the dimension of the enclosed spaces.}
\item{Specify where the presentation will take place.}
\item{Specify which space will be occupied by which robot.}
\end{itemize}

\subsection{Score Sheet}
The maximum time for this test is 60 minutes.

Robots are scored on functionality and on design.
The audience can awards tokens for what they elect to be the \textbf{Most functional robot} and the \textbf{Best looking robot}.

\begin{tabularx}{\textwidth}{ X r }
	\textbf{Action} & \textbf{Score} \\ \hline
	Maximum score won by most ``Functionality''-tokens & 2.5\\
	``Functionality''-tokens awarded to winning team.& \\
	``Functionality''-tokens awarded to this team. & \\
	\\
	Maximum score won by most ``Design''-tokens & 2.5\\
	``Design''-tokens awarded to winning team.& \\
	``Design''-tokens awarded to this team. & \\
	\\
	% \textbi{Special penalties & bonuses} & \\ %%%TODO: Cannot get `make` this line somehow?
	Not attending \ref{rule:not_attending} & -5\\
	Outstanding performance \ref{rule:outstanding_performance} & 1 \\
	\\ \hline
	\textbi{Total score (excluding penalties and bonuses)} & \textbf{5}\\
	\\ 
\end{tabularx}

\textbf{Normalization}: The teams with less tokens than the best team get proportional scores based
on the number of tokens they received, e.g 
$\text{score for this team} = 2.5 \times \frac{t_{this}}{t_{best}}$
where $t_{this}$, $t_{best}$ is the number of tokens received by this team, and the number of tokens received by the best team.
