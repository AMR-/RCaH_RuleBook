\section{Person Recognition}

An Operator is introduced to the robot, which needs to learn what the Operator looks like. Once the robot has gathered enough information about the Operator, the Operator mixes within a crowd and the robot needs to find the Operator. Once the robot has found its Operator, it must state some information about the Operator and the crowd, such as mood and genders.

\subsection{Goal}
The robot has to identify the Operator within a crowd and state information about the Operator and the crowd.

\subsection{Focus}

This test focuses on people detection and recognition; as well as pose recognition and human-robot interaction with unknown people.

\subsection{Setup}

\begin{enumerate}
\item \textbf{Operator:} A \quotes{professional} operator is selected by the TC to test the robot. 
  This person may a different be drafted from the crowd in each run.
\item \textbf{Other people} There are no restrictions on other people walking by or standing around throughout the complete task.
\end{enumerate}

\subsection{Task}
\paragraph{This test may also be held outside the arena
  This is in order to have the possibility to run multiple robots in parallel and reduce the total time needed to test all robots.}

\begin{enumerate}

\item \textbf{Start:} The robot starts at a designated starting position, and waits for the \quotes{professional} operator. 
  When the referees start the time, the team is not allowed to instruct the operator.
\item \textbf{Memorizing the operator:} The robot has to memorize the operator. 
  During this phase, the robot may instruct the operator to follow a certain setup procedure.
\item \textbf{Learning operator name:} Optionally, the robot may ask the operator for his/her name and make the interaction after finding the operator again more natural.
\item \textbf{Wait for Start Command:} Once the robot states it has finished memorizing the operator, 
  it must wait for the operator to mix with the crowd. The robot may assume that this is done in ca. 10 seconds. 
  Alternatively, it may wait for a Start Command via ASR (or using the Continue rule if need be; Section \refsec{rule:asrcontinue}) while the operator walks around the robot and moves behind it to blend in with the crowd.
  This test is not concerned with audio and voice recognition. Therefore, the start command may also be given by a single key press.
\item \textbf{Find the crowd:} After the time elapses, the robot must turn circa $180\degree$, approach the crowd and start looking for the operator.
\begin{itemize}
\item \textbf{Crowd size:} The crowd may contain between 5 and 10 people, standing or sitting or lying within an  area of 5 meters (diameter).
\item \textbf{Crowd position:} The crowd will be located behind the robot at a distance between 2 and 3 meters apart.
\end{itemize}
\item \textbf{Find the operator:} Once the crowd has been located, the robot must greet the operator (optionally by name) and state the gender, and pose (sitting, standing, raised arm(s)).
  Also, it must point or approach to the operator. In any case, it must be clear to the referee that the robot has found the operator, unambiguously.
  If this is not the case, no points will be scored. 

  For example: when the robot says \textit{I found you operator, you are the smiling girl sitting in the middle of the crowd.}
  In the case of two smiling girls in the middle of a four person crowd, there is ambiguity and thus no points.

\item \textbf{Describe the crowd}: Finally, robot must tell the size of the crowd and how many men, women and even children are.
\end{enumerate}

\subsection{Additional rules and remarks}
\begin{enumerate}
\item \textbf{Recognition report:} 
  Robots must create a PDF report file.
  This file may be stored on a USB-stick on the robot which is given to the TC after the test. 
  The PDF file name should include the team name and a timestamp. 
  Furthermore, it must be unmistakeable which information belongs to which person. 
  Persons must also be recognizable in the report by a human (TC) so that it can be scored. 
  An overview of the crowd with bounding boxes and labels attached to the bounding boxes is handy for the TC to score.
  False positives in the report (labeling an object which is not a person but e.g. a plant) are penalized.
  The report must include:
  \begin{compactitem}
    \item Picture of the operator during training, stating the gender.
    \item For each person in the crowd: gender, pose and whether the labeled person is the operator.
  \end{compactitem}

\item \textbf{Preparation:} The robot needs to wait for at least 1 min before the operator appears in front of the robot. During this waiting time the team is not allowed to touch the robot.
\item \textbf{Disturbances from outside:} If a person from the audience (severely) interferes with the robot in a way that makes it impossible to solve the task, the team may repeat the test immediately.
\item \textbf{Instruction:} The robot interacts with the operator, not the team. That is, the team is not allowed to instruct the operator.
\item \textbf{Children:} Note that the operator and crowd may be members of the audience as well, which may include children. A robot only looking up may look over a child operator. 
\end{enumerate}

\subsection{Data recording}
  Please record the following data (See \refsec{rule:datarecording}):
  \begin{itemize}
   \item Images These will not be publicly available due to privacy concerns of the operator and crowd. 
  \end{itemize}


%\subsubsection{Referee instructions}
%
%The referee needs to
%\begin{itemize}
%\item 
%\item 
%\end{itemize}

\subsection{OC instructions}

\textbf{2 hours before the test}
\begin{itemize}
\item Select the \quotes{professional} operator(s).
\item Select the crowd.
\end{itemize}

\textbf{During the test}
\begin{itemize}
\item Check safe operation of the robot; the robot needs to be stopped immediately if a person is going to be touched by the robot
\end{itemize}

\subsection{Score sheet}
The maximum time for this test is 5 minutes.

\begin{tabularx}{\textwidth}{X r c c c }

	\textbf{Action} & \textbf{Score} & \textbf{$1^{st}$ try} & \textbf{$2^{nd}$ try} & \textbf{$3^{rd}$ try} \\ \hline
	& & & & \\ 
	\textit{\textbf{Operator}} \\
	Approach or point at the operator & $30$ & \hrulefill & \hrulefill & \hrulefill \\
	Correctly state operator's gender & $30$ & \hrulefill & \hrulefill & \hrulefill \\
	Correctly state operator's pose & $30$ & \hrulefill & \hrulefill & \hrulefill \\
	& & & & \\ 
	\textit{\textbf{Crowd}} \\
	Correctly state crowd's size & $20$ & \hrulefill & \hrulefill & \hrulefill \\
	Correctly state crowd's number of men & $20$ & \hrulefill & \hrulefill & \hrulefill \\
	Correctly state crowd's number of women & $20$ & \hrulefill & \hrulefill & \hrulefill \\ \hline
	& & & & \\ 
	\textit{\textbf{Score per try}} & $150$ & \hrulefill & \hrulefill & \hrulefill \\ 
	& & & & \\ 
	\textbf{Total Score} & $150$ & & & \\ \cline{3-5}

\end{tabularx}\\


% Local Variables:
% TeX-master: "Rulebook"
% End:


% Local Variables:
% TeX-master: "Rulebook"
% End:
