\chapter[EEGPSR in detail]{E\textsuperscript{2}GPSR in detail.}
\label{chap:eegpsr-appendix}

\section{Command Generation}
EEGPSR commands are generated randomly using the official [EE]GPSR Command Generator and grammars publicly available at https://github.com/kyordhel/GPSRCmdGen. The official [EE]GPSR Command Generator and the official grammars will be made available two months before the competition. However, teams must be aware that the categories, objects and other data is provided for testing purposes only.

For each command to be executed, the Team Leader must choose a Command Category. If the Team Leader knows \textit{a priori} that the robot won't be able to execute the generated command, is advised to inform the operator immediately in order to proceed with the next command, saving this way valuable time for the task execution.

\subsection{Random Category Selection}
The team leader may request \textbf{once} to the referee to give to the robot a command from a random category. Extra points are given if the robot is able to successfully execute the given command. The random category selection is a \textit{one-time} request.

\subsection{Mixing Categories}
The team leader may request to the Technical Committee to test the robot with commands involving abilities from two or more categories. Mixing categories must be requested to the TC two hours before the test, and once requested there is no step back. Extra points are given if the robot is able to successfully execute the given command.


\section{Command retrieval explained}
The robot has to show it has understood the given command by stating all the required information to accomplish the task. For this purpose, the robot may repeat the understood command and ask for confirmation. It is not required to repeat the command word by word; rephrasing the command is allowed. For instance, if the robot is instructed to \quotes{place a coke onto the tray}, the robot may either say: \textit{\quotes{You want me to place a coke on the tray. Is that correct?}} or \textit{\quotes{do you want me to deliver a coke to the tray?}}.

If The robot can't correctly recognize the given command, it is allowed to request the operator to repeat the command up to three times. After three failed attempts, a new command is generated. Th team may opt to use a custom operator or the Continue rule (Section Section 3.8.15).

When a robot has partially understood the command, it is allowed to ask the operator for additional information (e.g. \textit{\quotes{did you say apple juice or pineapple juice?}}).

%%%%%%%%%%%%%%%%%%%%%%%%%%%%%%%%%%%%%%%%%%%%%%%%%%%%%%%%%%%%%%%%%%%%%%%%%%%%%
%
% Categories explained
%
%%%%%%%%%%%%%%%%%%%%%%%%%%%%%%%%%%%%%%%%%%%%%%%%%%%%%%%%%%%%%%%%%%%%%%%%%%%%%
\section{Categories explained}
\label{sec:eegpsr-categories-explained}
This section explain each of the categories of the test and provides examples on how the abilities are scored.

It is important to remark that there is no script or predefined way to solve the tasks, being most of them of ambiguous nature. It is up to the team to choose how to solve each tasks accordingly with the robot's capabilities.


%%%%%%%%%%%%%%%%%%%%%%%%%%%%%%%%%%%%%%%%%%%%%%%%%%%%%%%%%%%%%%%%%%%%%%%%%%%%%
%
% Category I explained
%
%%%%%%%%%%%%%%%%%%%%%%%%%%%%%%%%%%%%%%%%%%%%%%%%%%%%%%%%%%%%%%%%%%%%%%%%%%%%%
\subsection{Category I: Advanced Manipulation}
\label{sec:eegpsr-category1-explained}
Tasks from this category require handling objects into small or narrow spaces, manipulate tools, buttons, panels, and doors; two-handed manipulation, or eye-hand coordination. 

\subsubsection{Task examples}
\begin{itemize}
	\item Grasping objects from a box.
	\item Placing objects into a microwave.
	\item Shutdown the TV using its remote control.
	\item Transporting a tray.
	\item Pouring cereal in a bowl.
	\item Opening a bottle (twist, uncap, etc.).
\end{itemize}

\subsubsection{Command examples}
\begin{itemize}
	\item Hand me a coke from the fridge (the coke is inside the fridge).
	\item Bring me some flakes in a bowl.
	\item Put this book into the drawer.
	\item Turn off the TV.
	\item Put all the beverages on the dinner table.
\end{itemize}



%%%%%%%%%%%%%%%%%%%%%%%%%%%%%%%%%%%%%%%%%%%%%%%%%%%%%%%%%%%%%%%%%%%%%%%%%%%%%
%
% Category II explained
%
%%%%%%%%%%%%%%%%%%%%%%%%%%%%%%%%%%%%%%%%%%%%%%%%%%%%%%%%%%%%%%%%%%%%%%%%%%%%%
\subsection{Category II: Advanced Object Recognition}
\label{sec:eegpsr-category2-explained}
Tasks from this category require describing unknown objects, recognize objects from description, identify occluded objects and from the distance.

\subsubsection{Task examples}
\begin{itemize}
	\item Counting objects in a shelf.
	\item Describing unknown objects.
	\item Finding object from far distance.
	\item Finding objects from a description.
	\item Infer unknown object's class (category, e.g. snacks) from features.
	\item Object detection and recognition of occluded or hidden objects (behind of, inside of, etc.).
\end{itemize}

\subsubsection{Command examples}
\begin{itemize}
	\item Bring me the biggest pill bottle from the kitchen counter.
	\item Bring me the bookcase's right-most object.
	\item Describe the objects on the drawer to me.
	\item Tell me how many red apples are in the basket on the kitchen table.
	\item Count the snacks in the shelf and tell me how many there are.
\end{itemize}


%%%%%%%%%%%%%%%%%%%%%%%%%%%%%%%%%%%%%%%%%%%%%%%%%%%%%%%%%%%%%%%%%%%%%%%%%%%%%
%
% Category III explained
%
%%%%%%%%%%%%%%%%%%%%%%%%%%%%%%%%%%%%%%%%%%%%%%%%%%%%%%%%%%%%%%%%%%%%%%%%%%%%%
\subsection{Category III: Navigation & People Tracking}
\label{sec:eegpsr-category3-explained}
Tasks from this category require following or guiding people in crowded environments or through narrow spaces. The navigation may take place either inside or outside the arena.

\subsubsection{Task examples}
\begin{itemize}
	\item Following a person inside an elevator.
	\item Guiding a person to the toilet.
	\item Going through a multitude while following or guiding a person without loosing them.
	\item Avoiding people crossing or standing by while guiding or following.
	\item Performing real time mapping and localization.
\end{itemize}

\subsubsection{Command examples}
\begin{itemize}
	\item Guide the person at the entrance to the kitchen.
	\item 
	\item Follow the person in front of you and go to the bedroom (operator will guide the robot outside the arena, so it will need to go back). 
\end{itemize}


%%%%%%%%%%%%%%%%%%%%%%%%%%%%%%%%%%%%%%%%%%%%%%%%%%%%%%%%%%%%%%%%%%%%%%%%%%%%%
%
% Category IV explained
%
%%%%%%%%%%%%%%%%%%%%%%%%%%%%%%%%%%%%%%%%%%%%%%%%%%%%%%%%%%%%%%%%%%%%%%%%%%%%%
\subsection{Category IV: People & Activity Recognition}
\label{sec:eegpsr-category4-explained}
Tasks from this category require memorizing a person's features, describing unknown people, recognize people from description, and being able to find people hiding or from the distance.

\subsubsection{Task examples}
\begin{itemize}
	\item Describing a person in certain specific location.
	\item Delivering objects to a person that matches the given description.
	\item Reporting number of people in a room matching given description.
	\item Finding people performing certain activity.
	\item Finding people whose face or body or partially occluded or not facing the robot.
\end{itemize}

\subsubsection{Command examples}
\begin{itemize}
	\item Describe the person at the door.
	\item Ask Joe to come here (Joe is sleeping in the sofa).
	\item Take this coke to the girl [in the living room] wearing a red sweater.
	\item Tell me how many standing people there are in the dining room.
	\item Go to the living room and follow the waving person.
	\item Tell me what John is doing (John is reading a book).
\end{itemize}

\subsubsection{Meeting new people}
Say the generated command is \textit{ask Joe to come here}, since the robot has no knowledge of who is Joe, it is expected to ask \quotes{\textit{how can I recognize Joe?}} Two answers are possible:
\begin{itemize}
	\item \textbi{Meet Joe:} The person named \textit{Joe} will stand in front of the robot and follow robot's (not team's) instructions for training. The robot must announce when it has completed memorizing that person before proceeding to execute the command.
	\item \textbi{Joe is the...} A description indicating how to recognize \textit{Joe} is given to the robot. Retrieved information must be confirmed.
\end{itemize}

\paragraph{Remark: Category III Overlap.} There may be given extra points for requesting further information about a named person and conducting a training or associating the name with the description.

\paragraph{Remark: Category IV Overlap.} Referees will use the same names for the same people, for which the robot may interact with the same person more than once. There are extra points if the robot can identify in a later command a previously learned name and successfully recognize that (same) person without training or asking for description (see score sheet).


%%%%%%%%%%%%%%%%%%%%%%%%%%%%%%%%%%%%%%%%%%%%%%%%%%%%%%%%%%%%%%%%%%%%%%%%%%%%%
%
% Category V explained
%
%%%%%%%%%%%%%%%%%%%%%%%%%%%%%%%%%%%%%%%%%%%%%%%%%%%%%%%%%%%%%%%%%%%%%%%%%%%%%
\subsection{Category V: Incomplete
% and Obfuscated
Information}
\label{sec:eegpsr-category5-explained}
In this category, the commands given do not include all the information necessary to accomplish the task
%, or the information is obfuscated and needs to be elucidated or deducted.

\subsubsection{Incomplete information}
The robot gets a command that does not include all the information necessary to accomplish the task. The actual commands will be under-specified by, for example:
\begin{itemize}
	\item only giving the class of the object (\quotes{bring me a drink}) or location (\quotes{guide me to the table}), and not the actual object or location, or
	\item not providing the location (or its class).
\end{itemize}

The robot can ask questions to retrieve the missing information about the task, but is not required to. In the questions the robot has to make clear what it has already understood, e.g., tell the operator that it has understood \textit{to bring a particular beverage can}, but not \textit{where the can is} located in the arena. The robot may also simply start searching.

% \subsubsection{Obfuscated information}
% The robot gets a command that requires further processing to extract or infer the information necessary to accomplish the task (e.g. co-reference resolution), for example:
% \begin{itemize}
% 	\item 
% 	\item 
% \end{itemize}
% 
% The robot can ask questions to retrieve the missing information about the task, but is not required to. In the questions the robot has to make clear what it has already understood, e.g., tell the operator that it has understood to bring a particular beverage can, but not where can is located in the arena. The robot may also simply start searching.

\subsubsection{Task examples}
Tasks from this category are the same of GPSR (see \refsec{chap:gpsr-appendix-cat1} and \refec{chap:gpsr-appendix-cat2}), it is how the robot is commanded what changes.

\subsubsection{Command examples}
\begin{itemize}
	\item Bring me a drink (unspecified which drink).
	\item Guide a person to the kitchen (unspecified where is that person).
	\item Bring some snacks to the table (unspecified which table).
% 	\item Find John and Ana at the living room. Tell her the time.
% 	\item Ask Mary at the sofa,
% 	\item 
% 	\item 
\end{itemize}



%%%%%%%%%%%%%%%%%%%%%%%%%%%%%%%%%%%%%%%%%%%%%%%%%%%%%%%%%%%%%%%%%%%%%%%%%%%%%
%
% Category VI explained
%
%%%%%%%%%%%%%%%%%%%%%%%%%%%%%%%%%%%%%%%%%%%%%%%%%%%%%%%%%%%%%%%%%%%%%%%%%%%%%
\subsection{Category VI: Erroneous Information}
\label{sec:eegpsr-category6-explained}
The robot gets a command that contains erroneous information. The robot should be able to realize such an error while trying to carry out the task, and try to carry on an alternative solution. If the robot is unable to solve the problem, it must go back to the operator, and clearly state \textit{why} it wasn't able to accomplish the task.

\subsubsection{Task examples}
Tasks from this category are very much like the ones in GPSR (see \refsec{chap:gpsr-appendix-cat1}, \refec{chap:gpsr-appendix-cat2} and \refec{chap:gpsr-appendix-cat3}).

\subsubsection{Command examples}
\begin{itemize}
	\item Bring me a coke from the fridge. \\
	The coke is on the kitchen table. \\

	\item Take Ana from the sofa to her bed. \\
	Ana is lying on the floor, unconscious, next to the sofa. \\
\end{itemize}

		\item \textbf{Category VII: Memory and Environmental Reasoning.} The command requires remembering previously executed tasks. The robot should be able to realize previously performed changes to the environment and, when unable to solve the task, it must go back to the operator and clearly state \text{why} it wasn't able to accomplish the task.


%%%%%%%%%%%%%%%%%%%%%%%%%%%%%%%%%%%%%%%%%%%%%%%%%%%%%%%%%%%%%%%%%%%%%%%%%%%%%
%
% Category VII explained
%
%%%%%%%%%%%%%%%%%%%%%%%%%%%%%%%%%%%%%%%%%%%%%%%%%%%%%%%%%%%%%%%%%%%%%%%%%%%%%
\subsection{Category VII: Memory and Environmental Reasoning}
\label{sec:eegpsr-category7-explained}
The robot gets a command that does not include all the information necessary to accomplish the task, assuming that the missing information is already known by the robot or can be (easily) obtained from the environment, for example:
\begin{itemize}
	\item Requesting to interact with an object or person (\quotes{bring a coke to Mary}) with which the robot has interacted before (the robot guided Mary to the bedroom).
	\item Requesting to observe the environment to gather the information required to accomplish the task.
	% \item Requesting to modify the environment to get what is needed to accomplish the task.
\end{itemize}

The robot can ask questions to retrieve the missing information about the task, but is supposed not to do so. In the questions the robot has to make clear what it has already understood, e.g., tell the operator that it has understood to bring a particular beverage can, but not where can is located in the arena. 

If the robot is unable to solve the problem, it must go back to the operator, and clearly state \textit{why} it wasn't able to accomplish the task.


\subsubsection{Task examples}
Tasks from this category are very much like the ones in GPSR (see \refsec{chap:gpsr-appendix-cat1}, \refec{chap:gpsr-appendix-cat2} and \refec{chap:gpsr-appendix-cat3}), it is how the robot is commanded what changes.

\subsubsection{Command examples}
\begin{itemize}
	\item Take the orange juice from the shelf and give it to Mary. \\
	The robot just guided Mary to the bedroom. \\

	\item Bring me the coke from the fridge. \\
	The operator already has the coke on their hand. \\

	\item Check which beverages there are in the counter and offer one to James at the sofa. \\
	Counter has milk, beer, crackers, and apples.
\end{itemize}



%%%%%%%%%%%%%%%%%%%%%%%%%%%%%%%%%%%%%%%%%%%%%%%%%%%%%%%%%%%%%%%%%%%%%%%%%%%%%
%
% Category VIII explained
%
%%%%%%%%%%%%%%%%%%%%%%%%%%%%%%%%%%%%%%%%%%%%%%%%%%%%%%%%%%%%%%%%%%%%%%%%%%%%%
\subsection{Category VIII: Three at once}
\label{sec:eegpsr-category8-explained}
Command from this category are composed of \textit{three simple actions}, which the robot has to show it has recognized. The robot may repeat the understood command and ask for confirmation. If it can't recognize the command correctly, it can also ask the speaker to repeat the complete command.

Tasks from this category are much alike the ones in GPSR (see \refsec{chap:gpsr-appendix-cat1} and \refec{chap:gpsr-appendix-cat2}), requiring to master basic skills. Since commands must be accomplished as quick as possible, in this category speed is the key.

\subsubsection{Command examples}
\begin{itemize}
	\item Go to the kitchen counter, take the coke, and bring it to me.
	\item Bring the chips to Mary at the sofa, tell the time and follow her.
	\item Find a person in the living room, guide them to the kitchen and follow them.
	\item Take the chips from the counter, find a person in the bedroom, and go to the entrance.
\end{itemize}

