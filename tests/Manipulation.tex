\section{Manipulation and object recognition}

The robot must reach a bookcase in which there are 10 objects at different shelves in the bookcase. 
The robot must then identify and grasp and identity 5 of those objects and put those into a new, easy-to-reach shelve that the team/robot may choose.

Optionally, the robot may open a little door or drawer for additional points. 

\subsection{Goal}
The robot has to identify, grasp and correctly place several objects at different heights or positions.
Opening drawers and/or doors is optional.

\subsection{Focus}
This test focuses on object detection, and manipulation; as well as object recognition.

\subsection{Setup}
\paragraph{This test may also be held outside the arena.
  This is in order to have the possibility to run multiple robots in parallel and reduce the total time needed to test all robots.}

\begin{enumerate}
\item \textbf{Location:} One of the bookcases in or around the apartment is used for this test. The robot will start at a random distance between 1.0m and 1.5m from the bookcase.
The bookcase has at least 5 shelves between 0.30m and 1.80m from the ground. One of the shelves is empty or will be made empty when the team chooses a shelve.
\item \textbf{Objects:} The bookcase contains 10 objects from the Scenario Objects \ref{rule:scenario_objects}.
  The robot must grasp 5 objects and identify at least \emph{those} 5 objects. 
\item \textbf{Object distribution:} The objects are located as follows:
\begin{enumerate}
\item Known object in an upper shelf.
\item Known object in a middle shelf.
\item Known object in a lower shelf.
\item Alike object in a middle shelf.
\item Cloth/tray/bowl in a middle shelf
\item[Optional] An occluded or hidden object on a middle shelf (e.g.~behind another object or inside a bowl).
\end{enumerate}
\item \textbf{Door/drawer:} The bookcase contains a door and a drawer, in which an additional object is placed. Opening the door or drawer gives a bonus. 
Please note that may be more than one object in each shelf to fit all objects in.
\end{enumerate}

\subsection{Task}
\begin{enumerate}
\item \textbf{Searching for objects:} When told so by an operator, the robot approaches to the shelf from its nearby starting position and starts searching for objects.
\item \textbf{Grasping objects:} Any object found by the robot may be grasped by it. Before or right after grasping the object, the robot has to announce which object it has found.
\item \textbf{Placing objects:} After grasping the object, the robot has to safely place it (Section \refsec{rule:scenario_objects}) on the empty shelf a the middle of the bookcase. 
  The object must stay there for at least 10sec.
\item \textbf{Handling objects multiple times:} Scores can only be gained a single time for each specific object.
\item \textbf{Optional: Opening door or drawer:} The robot may open the door and/or drawer in the bookcase. An additional item is is located behind/in it. 
\end{enumerate}

\subsection{Additional rules and remarks}
\begin{enumerate}
\item \textbf{No setup:} The robot must be ready to start the test with a voice command or start button when requested by the referee. There is no setup time.
\item \textbf{Startup:} The robot must be started with a single voice command or via a start button (Section \refsec{rule:start_signal}). If the robot is unable to start it must be removed immediately.
\item \textbf{Single try:} The robot must be able to start from the first attempt. 
`There is no restart for this test. If the robot is unable to start it must be removed immediately.
\item \textbf{Collisions:} Slightly touching the shelves or the bookcase is tolerated. 
  Driving over the objects or any other form of a major collision is not allowed, and the referees directly stop the robot (Section \refsec{rule:safetyfirst}).
\item \textbf{Object types:} The objects selected from the \textit{Standard Objects Set} will be chosen to be easily detectable and contrasting with the shelf (ex.~red or black objects on a white shelf).
\item \textbf{Recognition report:} Robots must create a PDF report file including the list of recognized objects with a picture showing the object and the object name/label.
  This file may be stored on a USB-stick on the robot which is given to the TC after the test. The PDF file name should include the team name and a timestamp. 
  Furthermore, it must be unmistakeable which label belongs to which object. Objects must also be recognizable in the report by a human (TC) so that it can be scored. 
  An overview of the shelf with bounding boxes and labels attached to the bounding boxes is handy for the TC to score.
  False positives in the report (labeling an object which is not an object but e.g. the edge of the shelf) are penalized.
%\item \textbf{QR Codes:} The team may request to use a special set ob objects identified with QR codes if the robot is not able to correctly recognize the objects. 
  The use of this special QR-object-set must be announced to the TC at least on hour before the test starts. When QR Codes are used, no points are given for object recognition.
\end{enumerate}

\subsection{Referee instructions}

The referee needs to
\begin{itemize}
\item Place the objects in the bookcase
\item Make sure there is one empty shelf in the middle of the bookcase. Ask the team which shelve they want to be empty.
\item Put an item in the drawer and behind the door and close them.
\end{itemize}

\subsection{Score sheet}

The maximum time for this test is 3 minutes.

\begin{scorelist}

	\scoreheading{Grasping objects}
	\scoreitem[5]{10}{Grasping any object (and successfully lifting it up to at least 5 cm for more than 10 second)}

	\scoreheading{Placing objects}
	\scoreitem[5]{10}{Placing any object (safely and the objects stands still for more than 10 second)}

	\scoreheading{Recognizing objects}
	\scoreitem[5]{10}{Every correctly recognized object in the report file}
 	\scoreitem[5]{-5}{False positives (labeling non-objects like an edge of the shelf) in the report file}

	\scoreheading{Hidden object optional (up to 50 points)}
	\scoreitem{50}{Finding a hidden or occluded object}

	%\setTotalScore{200}
\end{scorelist}


\footnotetext{The minimum and maximum distance of the objects in the shelf is still being discussed. This value may change. }


% Local Variables:
% TeX-master: "Rulebook"
% End:


% Local Variables:
% TeX-master: "Rulebook"
% End:
