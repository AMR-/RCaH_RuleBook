\newcommand{\bonusRobotCoop}{50~}

\section{Open Challenge}
\label{sec:test_open_challenge}

During the Open Challenge teams are encouraged to demonstrate recent research results and the best of the robots' abilities. It focuses on the demonstration of new approaches/applications, human-robot interaction and scientific value.

\subsection{Task}

The Open Challenge consists of a demonstration and an interview part. 
It is an open demonstration which means that the teams may demonstrate anything they like.  
The performance of the teams is evaluated by a jury consisting of all team leaders, TC and EC.
\OpenDemonstrationTask{seven}{three}

\subsection{Presentation} 
During the demonstration, the team can present the addressed problem and the demonstrated approach.
\begin{itemize}
\item A video projector or screen, if available, may be used to present a brief (max. 1 minute) introduction to what will be shown. 
\item The team can also visualize robot's internals, e.g., percepts. 
\end{itemize}

It is important to note that the jury may decide to end the demonstration if there is nothing happening or nothing \emph{new} is happening.

\OpenDemonstrationChanges

\subsection{Jury evaluation}
\begin{enumerate}
  {\bf\item Jury of team leaders:} All teams have to provide \emph{one} person 
  (preferably the team-leader) to follow and evaluate the entire Open Challenge.
  {\bf\item Evaluation:} Both the demonstration of the robot(s), and the answers of the team in the interview part are evaluated.\\ 
  For each of the following \emph{evaluation criteria}, a maximum of \scoring{10 points} is given per jury member:
  \begin{enumerate}
  \item Overall demonstration
  \item Human-robot interaction in the demonstration
  \item Robot autonomy in the demonstration
  \item Realism and \emph{usefulness for daily life} (Can this robot become a product?)
  \item Novelty and (scientific) contribution (+contribution to the community)
  \item Difficulty and success of the demonstration 
  \end{enumerate}
  A jury member is not allowed to evaluate and give points for the own team.
  {\bf\item Normalization and outliers}: 
  \begin{enumerate}
    \item The points given by each jury member are scaled to obtain a maximum of \scoring{250 points} (i.e., multiplied by $\nicefrac{25}{6}$). 
    \item The total score for each team is the mean of the jury member scores.
      To neglect outliers, the $N$ best and worst scores are left out:
      $$\mbox{score} = \frac{\sum\mbox{team-leader-score}}{\mbox{number-of-teams} - (2N+1)},
      \quad N=\begin{cases}2, & \mbox{number-of-teams} \ge 10\\1, & \mbox{number-of-teams} < 10 \end{cases}$$
    \end{enumerate}
\end{enumerate}

\subsection{Additional rules and remarks}
\begin{enumerate}
	\item \textbf{Start signal:} There is no standard start-signal for this test.
	\item \textbf{Abort on request:} At any time during the demonstration, the jury may interrupt and abort the demonstration:
	\begin{enumerate}
		\item if nothing is shown: in case of longer delays (more than one minute), e.g., when the robot does not start or when it got stuck;
		\item if nothing new is shown: the demonstrated abilities were already shown in previous tests (to avoid dull demonstrations and push teams to present novel ideas).
	\end{enumerate}

	\item \textbf{Team-team-interaction:}  An extra bonus of up to \bonusRobotCoop points can be earned if robots from two teams (4 robots maximum, 2 from each team) successfully collaborate (robot-robot interaction).
	\begin{enumerate}
		\item This bonus is earned for both teams.
		\item The robot(s) of the other team must only play a minor role in the total demonstration.
		\item It must be made clear that the demonstrations from the two teams are not similar, otherwise the points cannot be awarded.
		\item In case a team receives two (or more) bonuses, the maximum bonus will be taken.
	\end{enumerate}
\end{enumerate}


% Local Variables:
% TeX-master: "Rulebook"
% End:
