%%%%%%%%%%%%%%%%%%%%%%%%%%%%%%%%%%%%%%%%%%%%%%%%%%%%%%%%%%%%%%%%%%%%%%%%%%%%%
%
% EEGPSR
%
%%%%%%%%%%%%%%%%%%%%%%%%%%%%%%%%%%%%%%%%%%%%%%%%%%%%%%%%%%%%%%%%%%%%%%%%%%%%%

% Number of concurrent teams
\newcommand{\eegpsrTeams}{2~}
% Maximum number of commands to be given to a robot
\newcommand{\eegpsrMaxCmd}{3~}
% Maximum amount of time given to a team to perform a single command
\newcommand{\eegpsrMaxCmdTime}{5~}
% Maximum amount of time given to a team to perform all commands
\newcommand{\eegpsrMaxTeamTime}{\eegpsrMaxCmd$\times$\eegpsrMaxCmdTime}

% \section[EEGPSR]{E\textsuperscript{2}GPSR \\ \normalsize{(Enhanced Endurance General Purpose Service Robot)}}
\section[EEGPSR]{Enhanced Endurance General Purpose Service Robot}
\label{sec:eegpsr}

%
% MAURICIO @2017
% Short instructions based on GPSR
%
This test evaluates the abilities of the robot that are required throughout the set of tests in stage II of this and previous years' RuleBooks. In this test the robot has to solve multiple tasks upon request over an extended period of time (30-45 minutes). That is, the test is not incorporated into a (predefined) story and there is neither a predefined order of tasks nor a predefined set of actions. The actions that are to be carried out by the robot are chosen randomly by the referees from a larger set of actions. These actions are organized in several categories targeting an special ability. Scoring depends on the abilities shown.

\subsection{Focus}
This test particularly focuses on the following aspects:
\begin{itemize}
	\item No predefined order of actions to carry out (to get away from state machine-like behavior programming).
	\item Increased complexity in speech recognition.
	\item Environmental (high-level) reasoning.
	\item Robust long-term operation.
\end{itemize}


\subsection{Task}

\begin{enumerate}
	\item \textbf{Entering and command retrieval:} The robot enters the arena and drives to a designated position where it has to wait for further commands. \\

	\item \textbf{Command generation:} A command is generated randomly, depending on the command category chosen by the team (see below). \\

	\begin{enumerate}
		\item \textbf{Category I: Advanced Manipulation.} The task requires handling objects into small or narrow spaces, manipulate tools, buttons, panels, and doors; two-handed manipulation, or eye-hand coordination.

		\item \textbf{Category II: Advanced Object Recognition.} The task requires describing unknown objects, recognize objects from description, identify occluded objects and from the distance.

		\item \textbf{Category III: Navigation \& People Tracking.} The task involves following or guiding people in crowded environments or through narrow spaces. The navigation may take place either inside or outside the arena.

		\item \textbf{Category IV: People \& Activity Recognition.} The task requires memorizing a person's features, describing unknown people, recognize people from description, and being able to find people hiding or from the distance.

		% \item \textbf{Category III: Incomplete and Obfuscated Information.} The robot gets a command that does not include all the information necessary to accomplish the task, or the information is obfuscated and needs to be elucidated or deducted.

		\item \textbf{Category V: Incomplete Information.} The robot gets a command that does not include all the information necessary to accomplish the task.

		\item \textbf{Category VI: Erroneous information.} The command contains erroneous information. The robot should be able to realize what went wrong and, when to carry on an alternative solution, it must go back to the operator and clearly state \text{why} it wasn't able to accomplish the task.

		\item \textbf{Category VII: Memory and Environmental Reasoning.} The command requires remembering previously executed tasks. The robot should be able to realize previously performed changes to the environment and, when unable to solve the task, it must go back to the operator and clearly state \text{why} it wasn't able to accomplish the task.

		\item \textbf{Category VIII: Three at once.} The command is composed by \textit{three simple actions}, which the robot has to show it has recognized. The robot may repeat the understood command and ask for confirmation. If it can't recognize the command correctly, it can also ask the speaker to repeat the complete command.
	\end{enumerate}

	\item \textbf{Task assignment:} The robot is given the command by the operator and may directly start to work on the task assignment.

	\item \textbf{Task execution:} The robot must stop the execution of a task and return to its designated position within \eegpsrMaxCmdTime minutes. Otherwise the robot must be moved to its designated position immediately. If a restart is still available to the team, it can be restarted at the designated position. \\

	\item \textbf{Returning:} After accomplishing the assigned task, the robot has to move back to its designated position to wait and retrieve the next command (i.e., go back to 1. without the need of re-entering the arena). \\
	% The robot can work on at most \eegpsrMaxCmd commands. \\

	\item \textbf{Timing:} The total time allotted to the robot for command retrieval and task execution is \eegpsrMaxTeamTime minutes. If the robot is not at its designated position after the time has expired, it must be moved at its designated position immediately.\\

\end{enumerate}

\subsection{Additional rules and remarks}
\label{sec:eegpsr-remarks}
\begin{enumerate}
	\item \textbf{CONTINUE rule:} Teams are able to use the CONTINUE rule in this test, with all the standard penalties it involves as described in section \refsec{rule:continue}.
	%The CONTINUE rule can only be used with the custom operator (e.g. both penalties of custom speaker and CONTINUE rule will be applied). 
	\\

	\item \textbf{Mixed or random category.} There are extra points if the robot is able to solve a command combining abilities from two or more categories, or if the team chooses the robot to solve a command from a random category. Mixing categories must be requested to the TC two hours before the test.\\

	\item \textbf{Number of Teams and Scheduling:} In each test slot, \eegpsrTeams teams may be competing in the arena concurrently. The robots will be tested in an interleaved fashion: The robots will retrieve commands and execute the task one after the other. As stated above, each robot will have a maximum amount of \eegpsrMaxCmdTime minutes per command (including time for retrieving the command and executing it). \\
	
	\item \textbf{Returning to designated position:} To facilitate a fluent and untroubled performance of the robots, they must return (or being returned) to their designated position before the \eegpsrMaxCmdTime minutes command time elapses. \textbf{If a robot moves from its designated position while another robot is working on a command, it must be immediately disabled} and moved to its designated position. If a restart is still available to the team, it can be restarted at its designated position. \\

	\item \textbf{Referees:} Since the score system in this test involves a subjective evaluation of the robot's behavior, the referees are EC/TC members. One referee is assigned to each team to judge performance, to measure the time for working on a command, and to keep track of the overall operating time of the robot. \\

	\item \textbf{Category selection:} For every of the three commands given to the robot, the team chooses the desired command category. Please do note that points for showing an ability can only be scored once.\\

	\item \textbf{Operator:}
	\begin{itemize}
		\item The person operating the robot is one of the referees (default operator).
		\item If the robot appears to consistently not be able to understand the operator, the referees ask the team to apply the CONTINUE rule (\refsec{rule:asrcontinue}).
	\end{itemize}

	\item \textbf{Inoperative robots:} If a robot gets stuck while trying to accomplish a task during a reasonable amount of time (e.g.~30 seconds), the referee may ask the team to move back the robot to its designated position, proceeding with the next robot. \\

	\item \textbf{Restart:} Robots will be restarted at their designated position (starting outside the arena is prohibited). If a robot requires a restart, the referee will proceed with the next robot.\\

	\item \textbf{Changing/Charging batteries:} The team may install a charging station at the designated position of the robot, if it does not hinder the other robots. However, the robot must connect itself with the charging station after carrying out a command. Changing batteries or manually connecting the robot with the charging station is allowed during a restart. \\

	\item \textbf{Scoring:} Robots are scored by successfully performed ability and full command completion within time. 
\end{enumerate}

\subsection{OC instructions}
\textbf{2h before test:}
\begin{itemize}
	\item Specify and announce the entrance/exit door for each robot. 
	\item Specify and announce the waiting position for each robot. 
\end{itemize}
\textbf{During the test:}
\begin{itemize}
	\item Help placing items and arranging people upon referee request.
\end{itemize}

\subsection{Referee instructions}
\textbf{During the test:}
\begin{itemize}
	\item Generate random sentences. %by an automatic sentence generator.
	\item Take the command and total time per team.
\end{itemize}


\newpage
\subsection{Score sheet}
\ifEvaluationSheet{

{\LARGE\textbf{Given commands:}}\vspace{4mm}

\newcommand{\eegpsrsstrow}{
	\multicolumn{6}{c}{\vspace{5mm}~}  \\ \hline
	\multicolumn{6}{c}{\vspace{5mm}~}  \\ \hline
	Category: 1 2 3 4 5 6 7 \vspace{8mm} &
%	Restart? & Custom Operator? & Continue? & ASR attempts: 1 2 3 \\
	{\footnotesize Restart?} & {\footnotesize Custom Operator?} & {\footnotesize MAN Bypass?} & {\footnotesize ASR Bypass?} & {\footnotesize ASR attempts:} $\Box \Box \Box$ \\
}

\begin{table}[h]
\begin{tabularx}{\textwidth}{X r r r r r}
	\textbf{\large Command 1:} & ~ & ~ & ~ & ~ & ~ \\ \hline
	\eegpsrsstrow

	\textbf{\large Command 1 $\cdot$ 2:} & ~ & ~ & ~ & ~ & ~ \\ \hline
	\eegpsrsstrow
	
	\textbf{\large Command 1 $\cdot$ 2 $\cdot$ 3 :} & ~ & ~ & ~ & ~ & ~ \\ \hline
	\eegpsrsstrow

	\textbf{\large Command 1 $\cdot$ 2 $\cdot$ 3 :} & ~ & ~ & ~ & ~ & ~ \\ \hline
	\eegpsrsstrow
\end{tabularx}
\end{table}
\vspace*{\fill}

\textbf{Remark: } Abilities marked with \textbf{*} are subjectively evaluated by  EC/TC members. Scoring is granted proportionally based on robot performance.

\newpage
}{}

The maximum time for this test is 40 minutes.

\begin{scorelist}
	\scoreheading{Performance}
	\scoreitem{15}{Understanding the command the $1^{st}$ attempt}
        \scoreitem{10}{Understanding the command the $2^{nd}$ attempt}
        \scoreitem{ 5}{Understanding the command the $3^{rd}$ attempt}
	\scoreitem{15}{Random category successfully solved}
	\scoreitem{20}{Mixing categories (bonus for each extra category)}
	
	\scoreheading{HRI}
	\scoreitem{ 5}{Answering a predefined question}
	\scoreitem{10}{Ask for missing information}
	\scoreitem{ 5}{Ask for command after detecting an event}
	\scoreitem{10}{Explain in detail why the robot could not accomplish a task *}
	\scoreitem{20}{Natural handover (give or take)}

	\scoreheading{Manipulation}
	\scoreitem{ 5}{Grab/place an object}
	\scoreitem{15}{Grab/place a stacked object}
	\scoreitem{30}{Manipulation in narrow spaces}
	\scoreitem{50}{Open/close a bottle/can *}
	\scoreitem{20}{Open/close a door/drawer *}
	\scoreitem{30}{Manipulation of buttons/levers/panels}
	\scoreitem{30}{Manipulation of tiny/heavy/slippery objects}
	\scoreitem{50}{Pour into a bowl *}
	\scoreitem{30}{Two-handed manipulation *}
	
	\scoreheading{Memory \& Awareness}
	\scoreitem{10}{Detect an expected event (within a reasonable amount of time) *}
	\scoreitem{20}{Detecting an unexpected event*}
	\scoreitem{20}{Provide information about changes in the environment and/or given commands*}

	\scoreheading{Navigation}
	\scoreitem{20}{Follow operator until stopped}
	\scoreitem{20}{Guide a human to location without loosing him or colliding}
	
	\scoreheading{Object recognition}
	\scoreitem{10}{Counting overall objects}
	\scoreitem{30}{Counting objects in category}
	\scoreitem{50}{Counting objects matching description *}
	\scoreitem{30}{Describing an unknown object}
	\scoreitem{30}{Find (and grasp) an object from a description *}
	\scoreitem{30}{Find occluded object (>50\% occlusion)}
	\scoreitem{50}{Find hidden object (100\% occlusion)}
	\scoreitem{50}{Infer unknown object's class (category) from features}
	\scoreitem{15}{Recognize alike object}
	\scoreitem{ 5}{Recognize known object}

	\scoreheading{People, pose and activity recognition}
	\scoreitem{15}{Detect a calling/waving person}
	\scoreitem{15}{Find a person in a given room}
	\scoreitem{15}{Recognize a newly learned face correctly}
	\scoreitem{20}{State the gender of a person}
	\scoreitem{15}{State the number of people in a group}
	\scoreitem{20}{State the pose of a person *}

	\setTotalScore{0}
\end{scorelist}


% Local Variables:
% TeX-master: "Rulebook"
% End:


% Local Variables:
% TeX-master: "Rulebook"
% End:
 
