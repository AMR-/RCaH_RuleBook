\section{Enhanced Endurance General Purpose Service Robot}

This test evaluates Human-Robot Interaction and the integration of advanced capabilities of two robots in parallel, for an extended amount of time. 
There is no predefined story and there is neither a predefined order of tasks. 
The actions that are to be carried out by the robot are randomly generated by the referees, 
  based on a known set of actions that the robots must be able to perform.

A robot must show it has recognized as command and may repeat the understood command and ask for confirmation. 
If it can't recognize the command correctly, it can also ask the speaker to repeat the complete command, or ask for further information.

\subsection{Focus}
This test particularly focuses on the following aspects:
\begin{itemize}
	\item No predefined order of actions to carry out (to get away from state machine-like behavior programming).
	\item Increased complexity in speech recognition (possible commands are less restricted in both actions/operators and arguments/objects, 
	  commands can include multiple objects, e.g., \quotes{put the apple on the kitchen table})
	\item More advanced capabilities (e.g. describing objects, pouring)
	\item Responding to events
	\item Command dependencies. The robot may be asked to transport an item from A to B. 
	  In another command it may be asked to transport the same item to loction C. 
	  The robot should be smart enough to know thatthe item can be grabbed at location B.
\end{itemize}

\subsection{Task}

\begin{enumerate}
	\item \textbf{Entering and command retrieval:} The robot enters the arena and drives to a designated position where it has to wait for further commands.
	\item \textbf{Command generation:} A command is generated randomly, depending on the command category chosen by the team (see below).
	\item \textbf{Task assignment:} The robot is given a command by the operator and may directly start to work on the task assignment. 
	The robot must must prove it has understood the given command by repeating it (Please see the remarks about this in section~\ref{sec:eegpsr_remarks}).
	If a robot is unable to perform a command, it must say so. 
	\item \textbf{Finishing a command:} After finishing a task, the must wait for a new command.
	\item \textbf{Exiting the arena:} When commanded to do so, a robot may leave the arena. 
\end{enumerate}

\subsection{Commands and actions}


The robots must be able to perform the following actions:
\begin{itemize}
 \item All of the abilities in Stage I.
  \begin{enumerate}
   \item Following
   \item Collision free navigation
   \item Guide a human
   \item Grab item
   \item Place item
   \item Open a door/drawer
   \item Speech recogniton & understanding
   \item Learn and recognize new faces
   \item Infer gender
   \item Count people in a group
   \item Recognize objects
  \end{enumerate}
 \item Waking someone up
 \item Describing unknown objects at a given location
 \item Grab described object
 \item Pour into a bowl
 \item Take an item from a human using a natural handover
 \item Give an item to a human using a natural handover
 \item Answer a question
 \item Find a person in a given room
 \item Find and bring an object of a given type at a given location
 \item Bring an item to a given location
\end{itemize}


\subsection{Events}
At any time during the test, one of the events below may occur. 
Events will only occur in the field of view of the robot. 
This restriction will be loosened in future competitions. 

When an event occurs, the robot must ask the actor in the event what to do. 
\begin{enumerate}
 \item Someone falls on the ground
 \item Walking away
 \item Standing up
 \item Sitting down
 \item Waking up
 \item Waving
\end{enumerate}


\paragraph{Command examples}
\begin{enumerate}
 \item Go to the kitchen.
 \item Go to the bedroom and wake up James.
 \item Bring me the newpaper.
 \item Give James the newspaper.
 \item Pour the cereals in the bowl.
 \item Go to James and answer his question.
 \item Tell me what items are in the bookcase.
 \item Guide me to the kitchen.
 \item This is James. Remember him.
 \item Follow me.
 \item Tell me how many people are in the kitchen.
 \item Tell me if the guest in the hallway is a boy or girl.
\end{enumerate}

\subsection{Additional rules and remarks}
\label{sec:eegpsr_remarks}
\begin{enumerate}
  \item Robots are not scored per command but rather by subcommands and abilities successfully and correctly achieved (if necessary for an efficient execution of the command). 
    Achieving a complete command does yield a 10\% bonus. 
\end{enumerate}

\subsection{Referee and OC instructions}
\textbf{2h before test:}
\begin{itemize}
\item Specify and announce the entrance door for each robot. 
\item Specify and announce the waiting position for each robot. 
\end{itemize}
\textbf{During the test:}
\begin{itemize}
\item Generate random sentences by an automatic sentence generator
\end{itemize}

\newpage
\subsection{Score sheet}
\ifEvaluationSheet{

{\LARGE\textbf{Given commands:}}\vspace{4mm}

\newcommand{\eegpsrsstrow}{
	\multicolumn{6}{c}{\vspace{5mm}~}  \\ \hline
	\multicolumn{6}{c}{\vspace{5mm}~}  \\ \hline
	Category: 1 2 3 4 5 6 7 \vspace{8mm} &
%	Restart? & Custom Operator? & Continue? & ASR attempts: 1 2 3 \\
	{\footnotesize Restart?} & {\footnotesize Custom Operator?} & {\footnotesize MAN Bypass?} & {\footnotesize ASR Bypass?} & {\footnotesize ASR attempts:} $\Box \Box \Box$ \\
}

\begin{table}[h]
\begin{tabularx}{\textwidth}{X r r r r r}
	\textbf{\large Command 1:} & ~ & ~ & ~ & ~ & ~ \\ \hline
	\eegpsrsstrow

	\textbf{\large Command 1 $\cdot$ 2:} & ~ & ~ & ~ & ~ & ~ \\ \hline
	\eegpsrsstrow
	
	\textbf{\large Command 1 $\cdot$ 2 $\cdot$ 3 :} & ~ & ~ & ~ & ~ & ~ \\ \hline
	\eegpsrsstrow

	\textbf{\large Command 1 $\cdot$ 2 $\cdot$ 3 :} & ~ & ~ & ~ & ~ & ~ \\ \hline
	\eegpsrsstrow
\end{tabularx}
\end{table}
\vspace*{\fill}

\textbf{Remark: } Abilities marked with \textbf{*} are subjectively evaluated by  EC/TC members. Scoring is granted proportionally based on robot performance.

\newpage
}{}

The maximum time for this test is 40 minutes.

\begin{scorelist}
	\scoreheading{Performance}
	\scoreitem{15}{Understanding the command the $1^{st}$ attempt}
        \scoreitem{10}{Understanding the command the $2^{nd}$ attempt}
        \scoreitem{ 5}{Understanding the command the $3^{rd}$ attempt}
	\scoreitem{15}{Random category successfully solved}
	\scoreitem{20}{Mixing categories (bonus for each extra category)}
	
	\scoreheading{HRI}
	\scoreitem{ 5}{Answering a predefined question}
	\scoreitem{10}{Ask for missing information}
	\scoreitem{ 5}{Ask for command after detecting an event}
	\scoreitem{10}{Explain in detail why the robot could not accomplish a task *}
	\scoreitem{20}{Natural handover (give or take)}

	\scoreheading{Manipulation}
	\scoreitem{ 5}{Grab/place an object}
	\scoreitem{15}{Grab/place a stacked object}
	\scoreitem{30}{Manipulation in narrow spaces}
	\scoreitem{50}{Open/close a bottle/can *}
	\scoreitem{20}{Open/close a door/drawer *}
	\scoreitem{30}{Manipulation of buttons/levers/panels}
	\scoreitem{30}{Manipulation of tiny/heavy/slippery objects}
	\scoreitem{50}{Pour into a bowl *}
	\scoreitem{30}{Two-handed manipulation *}
	
	\scoreheading{Memory \& Awareness}
	\scoreitem{10}{Detect an expected event (within a reasonable amount of time) *}
	\scoreitem{20}{Detecting an unexpected event*}
	\scoreitem{20}{Provide information about changes in the environment and/or given commands*}

	\scoreheading{Navigation}
	\scoreitem{20}{Follow operator until stopped}
	\scoreitem{20}{Guide a human to location without loosing him or colliding}
	
	\scoreheading{Object recognition}
	\scoreitem{10}{Counting overall objects}
	\scoreitem{30}{Counting objects in category}
	\scoreitem{50}{Counting objects matching description *}
	\scoreitem{30}{Describing an unknown object}
	\scoreitem{30}{Find (and grasp) an object from a description *}
	\scoreitem{30}{Find occluded object (>50\% occlusion)}
	\scoreitem{50}{Find hidden object (100\% occlusion)}
	\scoreitem{50}{Infer unknown object's class (category) from features}
	\scoreitem{15}{Recognize alike object}
	\scoreitem{ 5}{Recognize known object}

	\scoreheading{People, pose and activity recognition}
	\scoreitem{15}{Detect a calling/waving person}
	\scoreitem{15}{Find a person in a given room}
	\scoreitem{15}{Recognize a newly learned face correctly}
	\scoreitem{20}{State the gender of a person}
	\scoreitem{15}{State the number of people in a group}
	\scoreitem{20}{State the pose of a person *}

	\setTotalScore{0}
\end{scorelist}


% Local Variables:
% TeX-master: "Rulebook"
% End:


% Local Variables:
% TeX-master: "Rulebook"
% End:
 
