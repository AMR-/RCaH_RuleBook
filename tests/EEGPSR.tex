%%%%%%%%%%%%%%%%%%%%%%%%%%%%%%%%%%%%%%%%%%%%%%%%%%%%%%%%%%%%%%%%%%%%%%%%%%%%%
%
% EEGPSR
%
%%%%%%%%%%%%%%%%%%%%%%%%%%%%%%%%%%%%%%%%%%%%%%%%%%%%%%%%%%%%%%%%%%%%%%%%%%%%%

% Number of concurrent teams
\newcommand{\eegpsrTeams}{2~}
% Maximum number of commands to be given to a robot
\newcommand{\eegpsrMaxCmd}{3~}
% Maximum amount of time given to a team to perform a single command
\newcommand{\eegpsrMaxCmdTime}{5~}
% Maximum amount of time given to a team to perform all commands
\newcommand{\eegpsrMaxTeamTime}{\eegpsrMaxCmd$\times$\eegpsrMaxCmdTime}

\section[EEGPSR]{E\textsuperscript{2}GPSR \\ \normalsize{(Enhanced Endurance General Purpose Service Robot)}}
\label{sec:eegpsr}

This test evaluates the required robot abilities throughout the Stage I \& II tests this (2016) and previous years' rulebook. In EEGPSR the robot has to solve multiple tasks that are chosen randomly by the referees from a larger set of actions, over an extended period of time (30-45 minutes). In other words, tasks are not incorporated into a (predefined) story and there is neither a predefined order of tasks nor a predefined set of actions.
The actions to be carried out by the robot are organized in several categories with equivalent complexity but targeting different abilities.
It is upto the teams to choose how to execute the command and solving the involved tasks according to robot capabilities.
Scoring thereby depends on the complexity of the abilities shown.

\subsection{Focus}
This test particularly focuses on the following aspects:
\begin{itemize}
	\item No predefined order of actions to carry out.
	\item Increased complexity in speech recognition.
	\item More advanced capabilities
	\item Environmental (high-level) reasoning
	\item Robust long-term operation.

\end{itemize}

\subsection{Task}

\begin{enumerate}
	\item \textbf{Entering and command retrieval:} The robot enters the arena and drives to a designated position where it has to wait for further commands. \\

	\item \textbf{Command generation:} A command is generated randomly depending on the category chosen by the team (see below). All commands are composed by up to three actions that the robot has to show it has recognized. The robot may repeat the understood command and ask for confirmation. If it can't recognize the command correctly, it can also ask the speaker to repeat the whole command again.

	\begin{enumerate}
		\item \textbf{Category I:} The command is focused in \textbf{Advanced Manipulation}.

		\item \textbf{Category II:} The command is focused in \textbf{Advanced Object Recognition}.

		\item \textbf{Category III:} The robot gets a command focused in Human-Robot Interaction (\textbf{HRI}) that does not include all the necessary information to accomplish the task.

		\item \textbf{Category IV:} The command is focused in \textbf{Memory and Awareness}. This category can only be chosen after the team has successfully accomplished another command.

		\item \textbf{Category V:} The command is focused in \textbf{People Recognition} and \textbf{Navigation}.

		\item \textbf{Category VI:} The command is focused in \textbf{simple tasks} involving Manipulation, Object Recognition, and Person Recognition.

	\end{enumerate}

%	 \\

%	Each new generated command may require information from the previously given ones. Also, the difficulty of the new commands will be increased with respect of the previously generated ones. \\

	\item \textbf{Task assignment:} The robot is given a command by the operator and may directly start to work on the task assignment. If a robot is unable to perform a command, it should get back to the operator, and clearly state \textbf{why} it wasn't able to accomplish the task. \\

%	 \\

	\item \textbf{Task execution:} The robot must stop the execution of a task and return to its designated position within \eegpsrMaxCmdTime minutes. Otherwise the robot must be moved to its designated position immediately. If a restart is still available to the team, it can be restarted at the designated position. \\

	\item \textbf{Returning:} After accomplishing the assigned task, the robot has to move back to its designated position to wait and retrieve the next command (i.e., go back to 1. without the need of re-entering the arena). The robot can work on at most \eegpsrMaxCmd commands. \\

	\item \textbf{Timing:} The total time allotted to the robot for command retrieval and task execution is \eegpsrMaxTeamTime minutes. If the robot is not at its designated position after the time has expired, it must be moved at its designated position immediately. See the section on scheduling below as well.\\

	\item \textbf{Exiting the arena:} When commanded to do so, a robot should leave the arena. \\

\end{enumerate}

\subsection{Additional rules and remarks}
\label{sec:eegpsr-remarks}
\begin{enumerate}
	\item \textbf{CONTINUE rule:} Teams are able to use the CONTINUE rule in this test, with all the standard penalties it involves as described in section \refsec{rule:asrcontinue}.
	%The CONTINUE rule can only be used with the custom operator (e.g. both penalties of custom speaker and CONTINUE rule will be applied). 
	\\

	\item \textbf{Number of Teams and Scheduling:} In each test slot multiple teams (preferably \eegpsrTeams teams) may be competing in the arena concurrently. The robots will be tested in an interleaved fashion: The robots will retrieve commands and execute the task one after the other. As stated above, each robot will have a maximum amount of \eegpsrMaxCmdTime minutes per command (including time for retrieving the command and executing it). \\
	
	\item \textbf{Returning to designated position:} To facilitate a fluent and untroubled performance of the robots, they must return (or being returned) to their designated position before the \eegpsrMaxCmdTime minutes command time elapses. \textbf{If a robot moves from its designated position while another robot is working on a command, it must be immediately disabled} and moved to its designated position. If a restart is still available to the team, it can be restarted at its designated position. \\

	\item \textbf{Carrying robots:}	To carry the robot, at most two team members are allowed in the arena, and the robot must be moved as quickly as possible. To start or restart the robot, at most one team member may operate the robot. The team members moving and operating the robots must leave the arena immediately after the robot is placed or started. \\

	\item \textbf{Referees:} Since the score system in this test involves a subjective evaluation of the robot's behavior, the referees are EC/TC members. One referee is assigned to each team to judge performance, to measure the time for working on a command, and to keep track of the overall operating time of the robot. \\

	\item \textbf{Category selection:} For every of the three commands given to the robot, the team chooses the desired command category. Please do note that points for showing an ability can only be scored once, as also detailed in the next point.\\

	\item \textbf{Scoring:} Points are scored per ability with the total score of the test being the sum of the points scored in each successfully demonstrated ability while solving the tasks (see score sheet). Abilities will be scored considering the best execution only (e.g.~successfully grasping scores for \textit{grasping}), with the single exception of collision-free navigation. \\

	\item \textbf{Operator:}
	\begin{itemize}
		\item The person operating the robot is one of the referees (default operator).
		\item If the robot appears to consistently not be able to understand the operator, the referees ask the team to apply the CONTINUE rule (\refsec{rule:asrcontinue}).
	\end{itemize}

	\item \textbf{Inoperative robots:} If a robot gets stuck while trying to accomplish a task during a reasonable amount of time (e.g.~30 seconds), the referee may ask the team to move back the robot to its designated position, proceeding with the next robot. \\

	\item \textbf{Restart:} The number of commands to be given to a robot is three regardless if the restart were used or not. If a restart is required during before the first half of the total time allowed for execute a command elapses, a new command will be generated for the robot to perform. If the first half of the time has elapsed, the team may proceed with the restart but no new command will be generated and the robot must wait for the remaining commands (if any). Robots will be restarted at their designated position, \textbf{it won't be allowed to start outside the arena.} \\

	\item \textbf{Changing/Charging batteries:} The team may install a charging station at the designated position of the robot, if it does not hinder the other robots. However, the robot must connect itself with the charging station after carrying out a command. Changing batteries or manually connecting the robot with the charging station is allowed during a restart. \\

	\item \textbf{Scoring:} Robots are scored by successfully performed ability and full command completion within time. 
\end{enumerate}

\subsection{OC instructions}
\textbf{2h before test:}
\begin{itemize}
	\item Specify and announce the entrance/exit door for each robot. 
	\item Specify and announce the waiting position for each robot. 
\end{itemize}
\textbf{During the test:}
\begin{itemize}
	\item Help placing items and arranging people upon referee request.
\end{itemize}

\subsection{Referee instructions}
\textbf{During the test:}
\begin{itemize}
	\item Generate random sentences. %by an automatic sentence generator.
	\item Take the command and total time per team.
\end{itemize}


\newpage
\subsection{Score sheet}
\ifEvaluationSheet{

{\LARGE\textbf{Given commands:}}\vspace{4mm}

\newcommand{\eegpsrsstrow}{
	\multicolumn{6}{c}{\vspace{5mm}~}  \\ \hline
	\multicolumn{6}{c}{\vspace{5mm}~}  \\ \hline
	Category: 1 2 3 4 5 6 7 \vspace{8mm} &
%	Restart? & Custom Operator? & Continue? & ASR attempts: 1 2 3 \\
	{\footnotesize Restart?} & {\footnotesize Custom Operator?} & {\footnotesize MAN Bypass?} & {\footnotesize ASR Bypass?} & {\footnotesize ASR attempts:} $\Box \Box \Box$ \\
}

\begin{table}[h]
\begin{tabularx}{\textwidth}{X r r r r r}
	\textbf{\large Command 1:} & ~ & ~ & ~ & ~ & ~ \\ \hline
	\eegpsrsstrow

	\textbf{\large Command 1 $\cdot$ 2:} & ~ & ~ & ~ & ~ & ~ \\ \hline
	\eegpsrsstrow
	
	\textbf{\large Command 1 $\cdot$ 2 $\cdot$ 3 :} & ~ & ~ & ~ & ~ & ~ \\ \hline
	\eegpsrsstrow

	\textbf{\large Command 1 $\cdot$ 2 $\cdot$ 3 :} & ~ & ~ & ~ & ~ & ~ \\ \hline
	\eegpsrsstrow
\end{tabularx}
\end{table}
\vspace*{\fill}

\textbf{Remark: } Abilities marked with \textbf{*} are subjectively evaluated by  EC/TC members. Scoring is granted proportionally based on robot performance.

\newpage
}{}

The maximum time for this test is 40 minutes.

\begin{scorelist}
	\scoreheading{Performance}
	\scoreitem{15}{Understanding the command the $1^{st}$ attempt}
        \scoreitem{10}{Understanding the command the $2^{nd}$ attempt}
        \scoreitem{ 5}{Understanding the command the $3^{rd}$ attempt}
	\scoreitem{15}{Random category successfully solved}
	\scoreitem{20}{Mixing categories (bonus for each extra category)}
	
	\scoreheading{HRI}
	\scoreitem{ 5}{Answering a predefined question}
	\scoreitem{10}{Ask for missing information}
	\scoreitem{ 5}{Ask for command after detecting an event}
	\scoreitem{10}{Explain in detail why the robot could not accomplish a task *}
	\scoreitem{20}{Natural handover (give or take)}

	\scoreheading{Manipulation}
	\scoreitem{ 5}{Grab/place an object}
	\scoreitem{15}{Grab/place a stacked object}
	\scoreitem{30}{Manipulation in narrow spaces}
	\scoreitem{50}{Open/close a bottle/can *}
	\scoreitem{20}{Open/close a door/drawer *}
	\scoreitem{30}{Manipulation of buttons/levers/panels}
	\scoreitem{30}{Manipulation of tiny/heavy/slippery objects}
	\scoreitem{50}{Pour into a bowl *}
	\scoreitem{30}{Two-handed manipulation *}
	
	\scoreheading{Memory \& Awareness}
	\scoreitem{10}{Detect an expected event (within a reasonable amount of time) *}
	\scoreitem{20}{Detecting an unexpected event*}
	\scoreitem{20}{Provide information about changes in the environment and/or given commands*}

	\scoreheading{Navigation}
	\scoreitem{20}{Follow operator until stopped}
	\scoreitem{20}{Guide a human to location without loosing him or colliding}
	
	\scoreheading{Object recognition}
	\scoreitem{10}{Counting overall objects}
	\scoreitem{30}{Counting objects in category}
	\scoreitem{50}{Counting objects matching description *}
	\scoreitem{30}{Describing an unknown object}
	\scoreitem{30}{Find (and grasp) an object from a description *}
	\scoreitem{30}{Find occluded object (>50\% occlusion)}
	\scoreitem{50}{Find hidden object (100\% occlusion)}
	\scoreitem{50}{Infer unknown object's class (category) from features}
	\scoreitem{15}{Recognize alike object}
	\scoreitem{ 5}{Recognize known object}

	\scoreheading{People, pose and activity recognition}
	\scoreitem{15}{Detect a calling/waving person}
	\scoreitem{15}{Find a person in a given room}
	\scoreitem{15}{Recognize a newly learned face correctly}
	\scoreitem{20}{State the gender of a person}
	\scoreitem{15}{State the number of people in a group}
	\scoreitem{20}{State the pose of a person *}

	\setTotalScore{0}
\end{scorelist}


% Local Variables:
% TeX-master: "Rulebook"
% End:


% Local Variables:
% TeX-master: "Rulebook"
% End:
 
