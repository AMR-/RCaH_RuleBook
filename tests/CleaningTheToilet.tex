\section{Filling the Dishwasher [DSPL \& OPL]}
The robot is able to remove dirt from a toilet using a brush or a sponge.

\subsection{Goal}
The robot has to correctly identify a dark spot over a white surface close to water and rub it with an appropriate tool until is removed.

\subsection{Focus}
This test focuses on spot recognition (substances, texture and color) as well as manipulation in narrow spaces using specialized tools.

\subsection{Setup}
\begin{enumerate}

	\item \textbf{Location:} This test takes place in the bathroom. 
	\item \textbf{Start position:} The robot starts outside the bathroom next to the door.
	\item \textbf{Toilet:} The toilet is partially full with water. Close to the water surface there is an artificial spot (chocolate syrup, peanut butter, jam, etc). There is no water in the deposit.
	\item \textbf{Tool:} A standard toilet-cleaning brush or sponge is located next to the toilet.
\end{enumerate}


\subsection{Task}
The robot must execute the following tasks in any logical order, aiming to remove the spot.
\begin{itemize}
	\item \textbf{Enter the bathroom:} The robot locates itself in the apartment and navigates inside the bathroom.
	
	\item \textbf{Identify the spot:} The robot inspects the toilet and finds the spot, pointing at it, and stating that requires to be cleaned.
	
	\item \textbf{Grasp the tool:} The robot locates and grasps the cleaning tool, lifting is from its base and placing the cleaner inside the toilet.
	
	\item \textbf{Clean the spot:} The robot makes contact with the spot and rubs it until it has disappeared.

	\item \textbf{Flush:} After cleaning the spot, the robot flushes the toilet, rinsing the tool.

	\item \textbf{Place the tool:} The robot returns the cleaning tool to its base.

	\item \textbf{Leave the bathroom:} After completing the cleaning, the robot leaves bathroom.
\end{itemize}


\subsection{Additional rules and remarks}
\begin{enumerate}
	\item \textbf{Bypassing Manipulation:} Bypassing object manipulation via the CONTINUE rule (Section \refsec{rule:mancontinue}) is not allowed during this test.
	\item \textbf{No setup:} There is no setup time.
	\item \textbf{Custom tool:} Teams are allowed to use their own customized cleaning tools for this task.
	\item \textbf{Tool handover:} If the robot is unable to find and grasp the tool, it may request to the referee to hand-over it.
	\item \textbf{Startup:} The robot can be started with a simple voice command or via a start button (Section \refsec{rule:start_signal}). 
	\item \textbf{Single try:} The robot must be able to start from the first attempt. There is no restart for this test. If the robot is unable to start it must be removed immediately.
	\item \textbf{Collisions:} Slightly touching the the toilet is tolerated (but not advised). Crushing the cleaning tool or any other form of a major collision terminates the test immediately (See section \refsec{rule:safetyfirst}). The same applies for spilling water outside.
\end{enumerate}

\subsection{Data recording}
Please record the following data (See \refsec{rule:datarecording}):
\begin{itemize}
	\item Images
	\item Plans
\end{itemize}

\subsection{OC instructions}

\textbf{2 hours before the test}
\begin{itemize}
	\item Announce the approximated startup location for robots.
	\item Fill the toilet with water to an average level.
\end{itemize}

\textbf{During the test}
\begin{itemize}
	\item Announce the approximated startup location for robots.
	\item Clean the toilet and cleaning tool when necessary.
\end{itemize}
\textbf{After the test}
\begin{itemize}
	\item Drain all the water from the toilet.
\end{itemize}

\subsection{Referee instructions}
The referee needs to
\begin{itemize}
	\item Place the spots in the toilet.
	\item Hand the tool to the robot when requested by it.
\end{itemize}


\newpage
\subsection{Score sheet}
The maximum time for this test is 5 minutes.

\begin{scorelist}
	\scoreheading{Entering}
	\scoreitem{10}{Enter the bathroom}

	\scoreheading{Cleaning}
	\scoreitem{25}{Grasp the tool and take it out of its container}
	% Total: 35
	\scoreitem{10}{Move the tool inside the toilet}
	% Total: 45
	\scoreitem{30}{Touch the spot with the tool}
	% Total: 75
	\scoreitem{40}{Partially remove the spot}
	% Total: --
	\scoreitem{90}{Completely remove the spot}
	% Total: 165
	\scoreitem{50}{Flush the toilet (by pressing button or lowering the lever)}
	% Total: 215
	\scoreitem{25}{Place the tool back in its container}
	% Total: 240

	\scoreheading{Leaving}
	\scoreitem{10}{Leave the bathroom}
	% Total: 250

	\setTotalScore{250}
\end{scorelist}


% Local Variables:
% TeX-master: "Rulebook"
% End:


% Local Variables:
% TeX-master: "Rulebook"
% End: