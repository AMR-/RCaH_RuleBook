\section{Set a table and clean it up}

Setting up a table for a meal and cleaning it up afterwards is one of the most repetitive tasks in a household environment. How wonderful it would be to have a robot doing it for us? This tasks aims at evaluating the capability of robots at this task.

The task comprises two phases: in phase~I, the robot is asked to set up the table, according to an optional variation given by the referee, and in the phase~II, the robot is supposed to cleanup the table, returning all placed objects to their original location, including detecting and cleaning spills with a cleaning cloth.

This task takes place in a single room, designated \emph{kitchen,} with a \emph{table,} where the relevant objects are stored, some of then on an open closet and other on a closed closet.

In phase~I, the operator requests the robot to setup the table for a meal. In this request, the operator may specify one variation of the meal, e.g., milk or coffee, bread or cookies. The robot should then set the table using items stored in various places of the kitchen, and afterwards it must clean it up, including small spills on the table. Some of the items are easly accessible (e.g., over the kitchen counter) while others may require the opening of doors (e.g., inside a cloret). Furthermore, some of these items may constrain the way they are handled by the robot (e.g., avoid pouring contents of a container). These items must be placed on a table reasonably following social conventions in terms of their positions on the table\footnote{Strict adherence to any social convention is out of the scope of the competition, and thus will not be evaluated.}. Midway during placement, the robot owner may change/adjust the position of some of the items; the robot is expected to handle the situation appropriately.

In phase~II, the operator requests the robot to clean up the table. This includes both returning the placed objects to their original location, and cleaning up dirt and spills using a cleaning cloth. This requires the robot to memorize the location the objects were originally recognized.


\subsection{Example}

The operator requests the robot to set the table for breakfast. Then, the robot replies asking whether the operator prefers bread or cookies, to which the operator replies prefering bread.  Several items are placed on the table: a plate, cutlery, a cup, napkins, a basket with bread, a cereal box, etc. The basket with bread, a napkin, and the cereal box are on the kitchen counter; the plate, the cutlery, and the cup are inside a closet. The robot starts moving these items, one by one, from their original position to the table. Some of these objects pose specific challenges to robot manipulation: the napkin is a flexible object, the bread basket contains bread, thus the robot must hold it with care, the items inside the closet require the closet door to be opened, and the cutlery is non-trivial to grasp. After the bread basket is placed on the table, the robot owner decides to move its place, so the positioning of remaining objects must have this in consideration. Then the robot will await an instruction by the operator to clean up the table. While taking his/her meal, the table gets dirty with a spill. The robot must detect the spill and clean it up using a cleaning cloth. Then, all of the other items are returned to their original locations. The task concludes once the table returns to its original state.

\subsection{Goal}

The robot has to move a list of relevant objects, with a possible variation stated by the operator, onto the a predefined table in the kitchen. Then it should return all placed items to their original locations and cleanup dirt ansd spills from the table.

\subsection{Focus}

This test focuses on HRI, semantic mapping, object perception and manipulation.

\subsection{Setup}

Half of the objects are placed on the kitchen counter, while the remaining ones inside a predefined closet, which is closed before the robot entering the arena. The table should be initially cleared of any objects. The robot will start at a predefined location, away from both the kitchen counter and the table. The team may optionally specify to the referees the variations supported by the robot, with at least two options. The referee then selects one of them randomly, not disclosing the choice.

\subsection{Task}
\label{sattu:task}

\begin{enumerate}
\item \textbf{Requesting the task:} The operator requests the robot to set up the table.
\item\label{sattu:s2} \textbf{Asking for variation:} (optional) The robot asks the operator for an option, to which the operator replies with the randomly selected (and undisclosed) option.
\item \textbf{Searching for objects:} The robot must detect which objects are missing on the table and search for them either on the kitchen counter on inside the closet.
\item \textbf{Grasping objects:} The robot must grasp any missing object and move it to the table.
\item \textbf{Placing objects:} Each object must be placed in a socially accepted position, not coliding with any object there.
\item \textbf{Changing objects position:} The operator will change the position of two objects on the table at any time, before the placement of the last one.
\item \textbf{Cleaning up the table:} After the meal, the operator requests the robot to clean up the table. The robot returns the objects to their original location, \textit{i.e.,} where they were found in the first place.
\item \textbf{Cleanup dirt and spills:} The robot must detect dirt and spills on the table and clean them up using a cleaning cloth.
\end{enumerate}

\subsection{Additional rules and remarks}
\label{sattu:add}
\begin{enumerate}
\item \textbf{No setup:} The robot must be ready to start the test with a voice command or start button when requested by the referee. There is no setup time.
\item \textbf{Startup:} The robot must be started with a single voice command or via a start button (Section \refsec{rule:start_signal}). If the robot is unable to start it must be removed immediately.
% \item \textbf{Single try:} The robot must be able to start from the first attempt. 
% `There is no restart for this test. If the robot is unable to start it must be removed immediately.
\item \textbf{Collisions:} Slightly touching the table.
  Driving over the objects or any other form of a major collision is not allowed, and the referees directly stop the robot (Section \refsec{rule:safetyfirst}).
\item\label{sattu:objs} \textbf{Object types:} The objects selected from the \textit{Standard Objects Set} will be chosen to be easily detectable and contrasting with the background (kitchen counter or closet). A maximum of 6 objects is considered for this task, where two of them is hard to grasp, e.g., cutlery, and another two must be handled upright, e.g., cup.
\item \textbf{Recognition report:} Robots must create a PDF report file including the list of recognized objects with a picture showing the object and the object name/label.
  This file may be stored on a USB-stick on the robot which is given to the TC after the test. The PDF file name should include the team name and a timestamp. 
  Furthermore, it must be unmistakeable which label belongs to which object. Objects must also be recognizable in the report by a human (TC) so that it can be scored. 
%  An overview of the shelf with bounding boxes and labels attached to the bounding boxes is handy for the TC to score.
False positives in the report (labeling an object which is not an object but e.g. the edge of the shelf) are penalized.
%\item \textbf{QR Codes:} The team may request to use a special set ob objects identified with QR codes if the robot is not able to correctly recognize the objects. The use of this special QR-object-set must be announced to the TC at least on hour before the test starts. When QR Codes are used, no points are given for object recognition.
\item \textbf{Clear area:} The robot may assume that there are no obstacles between the table, the kitchen counter, and the closet.
\item \textbf{Object list:} A total of 6 objects is considered, 3 of them considered easy to grasp (e.g., a cereal box, a cup, and a plate), while the remaining 3 hard to grasp (e.g., cutlery, napkins, and a basket with bread).
\item \textbf{Task variation:} The team may provide the referees a written set of options for setting up the table. These options must be written as possible answers to the robot question (step~\ref{sattu:s2} in section~\ref{sattu:task}). The correct execution of the specified variation should be clearly visible, e.g., choice of an object placed on the table.
\end{enumerate}

\subsection{Data recording}
  Please record the following data (See \refsec{rule:datarecording}):
  \begin{itemize}
   \item Images of recognized objects
   \item List of moved items
  \end{itemize}

\subsection{Referee instructions}

The referee needs to
\begin{itemize}
\item Clean up any remaining object on the table.
\item Place the objects on either the kitchen counter or inside the closet, half of them in each one of these two locations. Each one of these locations must contain at least one easy to grasp and one hard to grasp object.
\item Close the closet door.
\item Ask the team whether they implemented a meal variation, and if yes, choose randomly one of the options, not disclosing the choice.
\end{itemize}

\subsection{OC instructions}

The Organization must provide:
\begin{itemize}
\item the objects, as specified in item~\ref{sattu:objs} of section~\ref{sattu:add}
\item both an open closet and a closet with a door in the kitchen room
\end{itemize}


\subsection{Score sheet}

The maximum time for this test is \textbf{10 minutes}. A maximum of 6 objects is
considered in this score sheet. 3 of those are easy to grasp, 3 are difficult to grasp (for a robot)

\begin{scorelist}

  \scoreheading{Meal variation}
  \scoreitem[1]{10}{For asking for the meal variation and confirming the choice}

	\scoreheading{Grasping objects}
	\scoreitem[12]{10}{For each successful grasp of any object (lifting it up to at least 5 cm for more than 10 seconds)}
	\scoreitem[12]{15}{For each successful grasp of an hard to grasp object (lifting it up to at least 5 cm for more than 10 seconds)}
        % Grasping = (* 12 (+ 10 15)) = 300

	\scoreheading{Placing objects}
	\scoreitem[6]{10}{For each successful placement of any object anywhere on the table (safely stands still for more than 10 seconds)}
	\scoreitem[6]{20}{For each successful placement of an hard to grasp object anywhere on the table (safely stands still for more than 10 seconds)}
        \scoreitem[1]{30}{For appropriately executing the operator's choice}
	\scoreitem[6]{-5}{For each collision of an object with another
        one on the table}
        % Placing = (+ (* 6 (+ 10 20)) 30) = 210

      \scoreheading{Cleaning up the  table}
	\scoreitem[6]{10}{For each successful placement of any object to its original location}
	\scoreitem[6]{20}{For each successful placement of an hard to grasp object to its original location(safely stands still for more than 10 seconds)}
        \scoreitem[1]{40}{For successfully cleaning up dirt and spill on the table}
        % Cleaning = (+ (* 6 (+ 10 20)) 40) = 220
	
% 	\scoreheading{Total task}
% 	\scoreitem[5]{40}{Place known object near known object of same class}
% 	\scoreitem[5]{50}{Place unknown object near unknown object of same class}

        \scoreheading{Bonus}
        \scoreitem[1]{15}{Open the door without human help}

	% (+ 10 Grasping Placing Cleaning Bonus))
	\setTotalScore{745}
\end{scorelist}

% \subsection{Score examples} 

% TODO

% \begin{itemize}
%  \item Robot A fails at all manipulation attemps but makes an excelent report, with the 5 known objects labeled correctly and the 5 pairs of unknown objects correctly, will get 
% $5*10 + 5*15 = 125$ points. 
%  \item Robot B that fails to recognize anything but does move all the objects from the table to anywhere in the cupboard receives $5*10 + 5*10 = 100$ points.
%  \item Robot C grasps a single unknown item (``Cookies''), places it at the correct position near the other ``Cookies'' and also makes the same excelent report as robot A will get 
% 10 points for grasping, 10 foor placing anywhere and an additional 5 for placing at the right location, so 25 for that single object.
% The  total for robot C is then 150 points.
% \end{itemize} 


% Local Variables:
% TeX-master: "Rulebook"
% End:


% Local Variables:
% TeX-master: "Rulebook"
% End:
