\chapter{Finals}

The competition ends with the Finals on the last day, where the five teams with the highest total score compete.
The \iterm{Finals} are conducted as a final open demonstration.
This demonstration does not have to be different from the other open demonstrations---open challenge and demo challenge. 
It does not have to be the same either.

\section{Final Demonstration}
In the final demonstration, every team qualified for the Finals can choose freely what to demonstrate. 
The demonstration is evaluated by both a league-internal and a league-external jury.

\subsection{Task}
The procedure for the demonstration and the timing of slots is as follows:
\OpenDemonstrationTask{ten}{five}

\subsection{Evaluation and Score System}
The demonstration is evaluated by both a league-internal and a league-external jury.
The final score and ranking are determined by the two jury evaluations and by the previous performance (in Stages I and II) of the team.
\begin{enumerate}
\item\textbf{League-internal jury:} The league-internal jury is formed by the Executive Committee.
The evaluation of the league-internal jury is based on the following criteria:
  \begin{compactenum}
  \item Scientific contribution %(maybe taken from the OC)
  \item Contribution to @Home %(evaluated by Execs/TC)
  \item Relevance for @Home / Novelty of approaches %(evaluated by execs/TC)
  \item Presentation and performance in the finals.
  \end{compactenum}
It is expected that teams present their scientific and technical contributions in both \iterm{team description paper} and the \iterm{RoboCup\char64Home Wiki}.
In addition, finalist teams may provide a printed document to the jury (max 2 pages) that summarizes the demonstrated robot capabilities and contributions.  

  The influence of the league-internal jury to the final ranking is \SI{25}{\percent}.

\item \textbf{League-external jury:} The league-external jury consists of people not being involved in the RoboCup@Home league,
but having a related background (not necessarily robotics).
They are appointed by the Executive Committee.
The evaluation of the league-external jury is based on the following criteria:
  \begin{compactenum}
  \item Originality and Presentation
    (story-telling is to be rewarded)
  \item Usability / Human-robot interaction
  \item Multi-modality / System integration
  \item Difficulty and success of the performance
  \item Relevance / Usefulness for daily life
  \end{compactenum}
  The influence of the league-external jury to the final ranking is \SI{25}{\percent}.

\item \textbf{Previous performance:} \SI{50}{\percent} of the final score are determined by the team's previous performance during the competition, i.e., 
the sum of points scored in Stage I and Stage II.
\end{enumerate}

\OpenDemonstrationChanges

%% %%%%%%%%%%%%%%%%%%%%%%%%
\section{Final Ranking and Winner}

The winner of the competition is the team that gets the highest
ranking in the finals.

There will be an award for 1st, 2nd and 3rd place. All teams in the
Finals receive a certificate stating that they made it into the Finals
of the RoboCup@Home competition.


% Local Variables:
% TeX-master: "Rulebook"
% End:
