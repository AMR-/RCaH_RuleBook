\chapter{Finals}

The competition ends with the Finals on the last day, where the five teams with the highest total score compete. The \iterm{Finals} are conducted as a final open demonstration where the robots show their best abilities. This demonstration does not have to be different from the other open demonstration ---open challenge--- nor have to be the same either.

\section{Final Demonstration}

In the final demonstration, every team qualified for the finals can choose freely what to demonstrate. The demonstration is evaluated by both a league-internal and a league-external jury, considering also the score during the competition.

It is intended to show that a robot is able to perform a set of advanced skills integrated into a simple story driven at home. The story, in which the robot is the main character, must be easy to understand and self explicative.

\subsection{Task}
The procedure for the demonstration and the timing of slots is as follows:
\begin{enumerate}
  \item \textbf{Setup and demonstration:} The team has a maximum of ten minutes for setup and demonstration. During the demonstration, the robot must perform at least 2 complex tasks from different categories (see \refsec{chap:example-skills}for a list of examples on each category) to be evaluated by the League-internal jury. During Setup Time and before the demonstration begins, the team leader is allowed to \emph{very} briefly describe the story (maximum time is one minute). \\
  \item \textbf{Interview and cleanup:} After the demonstration, there is another five minutes where the team answers questions by the jury members. The team may prepare one slide with technical information of the task to rely on during the interview in case that projectors are available.

  During the interview time, the team has to undo its changes to the environment.
\end{enumerate}

\subsection{Evaluation and Score System}
The demonstration is evaluated by both a league-internal and a league-external jury. The final score and ranking are determined by the two jury evaluations and by the previous performance (in Stages I and II) of the team.

\begin{enumerate}
  \item \textbf{League-internal jury:} The league-internal jury is formed by the Executive Committee.
  The evaluation of the league-internal jury is based on the following criteria:
  \begin{enumerate}
    % \item Novelty (Seen before in @Home?)
    \item Scientific contribution (Is that new in @Home?)
    \item Performance executing complex skill 1
    \item Performance executing complex skill 2
    \item Contribution for @Home (can other teams use the solution?)
    \item Performance executing each additional complex skills (if any, scoring as bonus)

  % MAURICIO: Previous evaluation criteria (2014)
  %  \item Scientific contribution
  %  \item Contribution to @Home
  %  \item Relevance for @Home / Novelty of approaches
  %  \item Presentation and performance in the finals.
  \end{enumerate}
  It is expected that teams present their scientific and technical contributions in
  % MAURICIO: I removed the wiki
  % both team description paper and the RoboCup@Home Wiki.
  the team description paper .
  In addition, finalist teams may provide a printed document to the jury (max 2 pages) that summarizes the demonstrated robot capabilities and contributions.

  The influence of the league-internal jury to the final ranking is 25\%. \\

  \item \textbf{League-external jury:} The league-external jury consists of people not being involved in the RoboCup@Home league, but having a related background (not necessarily robotics). They are appointed by the Executive Committee. The evaluation of the league-external jury is based on the following criteria:
  \begin{enumerate}
    \item Integration of skills in story (story-telling is to be rewarded)
    \item Difficulty of the performance (How difficult is it?)
    \item Success of the performance (The robot did it?)
    \item System integration (How smooth was the execution?)
    \item Relevance / Usefulness for daily life (I want that robot in my home!)
  % MAURICIO: Previous evaluation criteria (2014)
  %  \item Originality and Presentation (story-telling is to be rewarded)
  %  \item Usability / Human-robot interaction
  %  \item Multi-modality / System integration
  %  \item Difficulty and success of the performance
  %  \item Relevance / Usefulness for daily life
  \end{enumerate}

  The influence of the league-external jury to the final ranking is 25\%. \\

  \item \textbf{Previous performance:} 50\% of the final score are determined by the team's previous performance during the competition, i.e., the sum of points scored in Stage I and Stage II.
\end{enumerate}

\subsection{Changes to the environment}
\begin{enumerate}
  \item Making changes: As in the other open demonstrations, teams are allowed to make modifications to the arena as they like, but under the condition that they are reversible.
  \item Undoing changes: In the interview and cleanup team, changes need to be made undone by the team. The team has to leave the arena in the very same condition they entered it.
\end{enumerate}

\subsection{Final Ranking and Winner}
The winner of the competition is the team that gets the highest ranking in the finals

There will be an award for 1st, 2nd and 3rd place. All teams in the Finals receive a certificate stating that they made it into the Finals of the RoboCup@Home competition.


% Local Variables:
% TeX-master: "Rulebook"
% End:
