\section{Filling the Dishwasher [DSPL \& OPL]}
The robot helps by placing all dirty dishes inside the dishwasher

\subsection{Goal}
The robot has to correctly identify and manipulate cutlery, kitchenware, and tableware, taking them from the sink and safely placing them in the dishwasher rack.

\subsection{Focus}
This test focuses on object recognition and manipulation, paying particular attention in orientating objects for an adequate placement in narrow spaces.

\subsection{Setup}
\begin{enumerate}
	\item \textbf{Dishwasher:} The dishwasher has the door open and its bottom rack half way outside, ready to be filled in.
	\item \textbf{Location:} This test takes place in the kitchen of the arena. 
	\item \textbf{Sink:} The sink has 10 utensils (See \ref{rule:tableware_kitchenware}) in it. There are at least two dishes (or bowls) two mugs (or glasses, or cups), one cooking pan (or pot), and one spoon (or any other cutlery).
	\item \textbf{Start position:} The robot will start somewhere between the \textit{sink} and the \textit{dishwasher}.
\end{enumerate}


\subsection{Task}
The robot must execute the following tasks in any order as long as there are utensils in the sink.
\begin{itemize}
	\item \textbf{Place cutlery:} The robot takes any cutlery from the sink and safely places it in the cutlery basket of the dishwasher.
	\item \textbf{Place cooking pans:} The robot takes any cooking pan (or pot, or utensil) from the sink and safely places it in the the dishwasher.
	\item \textbf{Place dishes and bowls:} The robot takes any dish (or bowl) from the sink and safely places it in the corresponding rack in the dishwasher.
	\item \textbf{Place glasses, mugs, and cups:} The robot takes any glass (or cup, or mug) from the sink and safely places it in the corresponding rack in the dishwasher.
\end{itemize}


\subsection{Additional rules and remarks}
\begin{enumerate}
	\item \textbf{Bypassing Manipulation:} Bypassing object manipulation via the CONTINUE rule (Section \refsec{rule:mancontinue}) is not allowed during this test.
	\item \textbf{No setup:} There is no setup time.
	\item \textbf{Startup:} The robot can be started with a simple voice command or via a start button (Section \refsec{rule:start_signal}). 
	\item \textbf{Single try:} The robot must be able to start from the first attempt. There is no restart for this test. If the robot is unable to start it must be removed immediately.
	\item \textbf{Collisions:} Slightly touching the the dishwasher is tolerated (but not advised). Crushing objects or any other form of a major collision terminates the test immediately (See section \refsec{rule:safetyfirst}).
	\item \textbf{Safely placing:} All objects must be allocated in a smart-way, maximizing the number of dishes to fit in, but without obstructing the functioning of the moving parts of the dishwasher (i.e., a human-like behavior is expected). For example, dishes should be placed vertically in the bottom tray, with pans lying horizontally, and glasses and mugs in the upper rack. It is understood that all liquid containers must be placed upside down.
\end{enumerate}

\subsection{Data recording}
Please record the following data (See \refsec{rule:datarecording}):
\begin{itemize}
	\item Images
	\item Plans
\end{itemize}

\subsection{OC instructions}

\textbf{2 hours before the test}
\begin{itemize}
    \item Announce the approximated startup location for robots.
    \item Announce the location of the sink.
\end{itemize}

\subsection{Referee instructions}
The referee needs to
\begin{itemize}
	\item Place the objects in the sink and a few of the same class on the dishwasher. 
\end{itemize}


\newpage
\subsection{Score sheet}
The maximum time for this test is 5 minutes.

\begin{scorelist}
	\scoreheading{Grasping}
	% 60pts
	\scoreitem[6]{10}{Manipulating an utensil outside the sink}
	% Total: 60

	% 100pts
	\scoreheading{Tableware}
	\scoreitem[2]{25}{Neatly placing a dish or bowl}
	\scoreitem[2]{25}{Neatly placing a cup, mug, or glass}
	% Total: 160

	% 50pts
	\scoreheading{Kitchenware}
	\scoreitem{40}{Neatly placing a cooking pan or pot}
	% Total: 200

	% 50pts
	\scoreheading{Cutlery}
	\scoreitem{50}{Neatly placing a knife, fork, or spoon in the basket}
	% Total: 250

	% 100pts
	\scoreheading{Additional utensils}
	\scoreitem[4]{25}{Each additional utensil neatly placed in the dishwasher}
	% Total: 350

	% Doesn't count
	\scoreheading{Incorrect placements}
	\scoreitem{10}{Incorrectly place an object in the dishwasher}
	\scoreitem[6]{-25}{Object obstructs dishwasher mechanical parts}

	\setTotalScore{350}
\end{scorelist}


% Local Variables:
% TeX-master: "Rulebook"
% End:


% Local Variables:
% TeX-master: "Rulebook"
% End: