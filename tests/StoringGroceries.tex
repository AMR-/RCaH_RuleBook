\section{Storing Groceries}

The robot must help put the newly bought groceries in the right place.
The owner of the robot will put the items on a table and the robot helps the owner by putting the groceries in the right place in the cupboard.
What the right place for an item is defined by the objects already in the cupboard: objects of the same type must be placed together.
For example, the new pack of cookies from the table must be placed by the almost empty pack cookies in the cupboard.

In the cupboard and on the table, there will be both known, alike and unknown objects. 

\subsection{Example}
For example, say the known objects include the classes ``Chips'', ``Apple'', ``Milk'', ``Coffee'' and ``Tea''. 
There are three unknown object classes: ``Pear'', ``Cookies'' and ``Butter''.
These class names are unknown to the robot, but a human will immediately infer the class names.
For the robot's recognition, it suffices to be able to give the ``Cookies'' all the same label, eg. ``label0''. 

\subsection{Goal}
The robot has to identify, grasp and correctly place several objects at different heights or positions.
To start the storing the groceries, the robot needs to open the cupboard. 

\subsection{Focus}
This test focuses on object detection, manipulation and object recognition.

\subsection{Setup}
In the arena, there will be a table and cupboard close together, where the robot does not have to spend a lot of time driving between the two. 

\begin{enumerate}
\item \textbf{Location:} One of the bookcases or cupboards in the apartment is used for this test, one where a table is near or can be put. 
The robot will start somewhere between the cupboard and the table. 
The cupboard has at least 5 shelves between 0.30m and 1.80m from the ground. 
One of the shelves is empty or will be made empty when the team chooses a shelve.
\item \textbf{Objects:} The cupboard contains 10 objects from the Scenario Objects \ref{rule:scenario_objects}.
\item \textbf{Door:} The cupboard has a single door, which is closed initially.
The robot may ask a human to open the door, after which a referee will open the door. 
\end{enumerate}

Please note that there may be more than one object in each shelf to fit all objects in.

\subsection{Task}
\begin{enumerate}
\item \textbf{Opening door or drawer:} The cupboard's door is closed and must be opened.
\item \textbf{Searching for objects:} When told so by an operator, the robot approaches the table from its nearby starting position and starts searching for objects. 
\item \textbf{Grasping objects:} Any object found on the table by the robot may be grasped by it. 
  Before or right after grasping the object, the robot may announce which object it has found. 
  % The scoring only takes the classifications in the report into account. 
\item \textbf{Placing objects:} After grasping the object, the robot has to safely place it (Section \refsec{rule:scenario_objects}) near the item of the same class in the cupboard. 
  The object must stay there for at least 10 seconds.
\end{enumerate}

\subsection{Additional rules and remarks}
\begin{enumerate}
\item \textbf{No setup:} The robot must be ready to start the test with a voice command or start button when requested by the referee. There is no setup time.
\item \textbf{Startup:} The robot must be started with a single voice command or via a start button (Section \refsec{rule:start_signal}). If the robot is unable to start it must be removed immediately.
\item \textbf{Single try:} The robot must be able to start from the first attempt. 
`There is no restart for this test. If the robot is unable to start it must be removed immediately.
\item \textbf{Collisions:} Slightly touching the the cupboard is tolerated.
  Driving over the objects or any other form of a major collision is not allowed, and the referees directly stop the robot (Section \refsec{rule:safetyfirst}).
\item \textbf{Object types:} The objects selected from the \textit{Standard Objects Set} will be chosen to be easily detectable and contrasting with the shelf (ex.~red or black objects on a white shelf).
\item \textbf{Recognition report:} Robots must create a PDF report file including the list of recognized objects with a picture showing the object and the object name/label.
  This file may be stored on a USB-stick on the robot which is given to the TC after the test. The PDF file name should include the team name and a timestamp. 
  Furthermore, it must be unmistakeable which label belongs to which object. Objects must also be recognizable in the report by a human (TC) so that it can be scored. 
  An overview of the shelf with bounding boxes and labels attached to the bounding boxes is handy for the TC to score.
  False positives in the report (labeling an object which is not an object but e.g. the edge of the shelf) are penalized.
%\item \textbf{QR Codes:} The team may request to use a special set ob objects identified with QR codes if the robot is not able to correctly recognize the objects. The use of this special QR-object-set must be announced to the TC at least on hour before the test starts. When QR Codes are used, no points are given for object recognition.
  \item \textbf{Clear area: } The robot may assume that the direct vicinity of the cupboard is clear and that the robot can move slightly backwards for its task. 
\end{enumerate}

\subsection{Data recording}
  Please record the following data (See \refsec{rule:datarecording}):
  \begin{itemize}
   \item Images
   \item Plans
  \end{itemize}

\subsection{Referee instructions}

The referee needs to
\begin{itemize}
\item Place the objects in the cupboard and a few of the same class on the table. New items can be placed on the table when the robot asks for that. 
\item Close the door of the cupboard. 
\item 5 of the objects in the cupboard must be unknown objects; 5 objects must be known or alike objects. 
\end{itemize}

\subsection{Score sheet}

The maximum time for this test is 3 minutes.

\begin{scorelist}
% There are 5 filled shelves
% On each shelf, there can fit 4 objects: 2 originally there, in the corners. As each object may get a 'companion' put there by the robot, there are 2 more objects per shelf.

% Grasp (any object): 10
% Place (anywhere in the cupboard): 10
% Place in correct place: 15
% Recognize known object correctly (without grasping/placing something of that class): 10
% Label two unknown objects of the same class with the same label (e.g. ``class0''): 15

% Place known object near known object of same class: 40
% Place unknown object near unknown object of the same class: 50


	\scoreheading{Grasping objects}
	\scoreitem[5]{10}{For each successful grasp of any object (lifting it up to at least 5 cm for more than 10 seconds)}

	\scoreheading{Placing objects}
	\scoreitem[5]{10}{For each successful placement of an object anywhere in the cupboard (safely stands still for more than 10 seconds)}
	\scoreitem[5]{5}{For each successful placement of an object at correct place (near an object of the same class)}

	\scoreheading{Recognizing objects}
	\scoreitem[5]{10}{Every correctly recognized known or alike object in the report file}
	\scoreitem[5]{15}{Corresponding labels for pairs of unknown objects of the same class, in the report file 
            \footnote{Suppose there is a rubber ducky as an unknown object class. Two rubber duckies must be given the same label (e.g. ``label0'') to receive these points}}
	\scoreitem[5]{-5}{False positive label}
	
% 	\scoreheading{Total task}
% 	\scoreitem[5]{40}{Place known object near known object of same class}
% 	\scoreitem[5]{50}{Place unknown object near unknown object of same class}

	\scoreheading{Bonus}
	\scoreitem[1]{20}{Open the door without human help}
	
	\setTotalScore{250}
\end{scorelist}


% Local Variables:
% TeX-master: "Rulebook"
% End:


% Local Variables:
% TeX-master: "Rulebook"
% End:
