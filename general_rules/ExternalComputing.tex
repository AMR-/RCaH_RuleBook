%%%%%%%%%%%%%%%%%%%%%%%%%%%%%%%%%%%%%%%%%%%%%%%%%%%%%%%%%
\section{External computing}\label{rule:robot_external_computing}
Robots are allowed to use some form of external computing, for example in the form of so-called ``Cloud services'' and/or ``Internet API's'' etc. 
\begin{enumerate}
	\item \textbf{Definition:} Computing not performed by a physical part (computer) of the robot are \iterm{external computing resources}. 
	\item \textbf{Inspection:} In general, external computers are not allowed unless explained to the \iaterm{Technical Committee}{TC}.
	  A team must announce to the TC at least 1 month in advance the external computing resources they want to use, for what purpose and how to reach the resources.
	  E.g. specify the URL or IP-address. 
	\item \textbf{Connection:} The robot may connect to \iaterm{external computing resources} via a network connection, e.g. the Internet. 
	  The competition organisation cannot guarantee that the connectivity of the connection. 
	\item \textbf{Autonomy:} The robot has to maintain full autonomy when using \iaterm{external computing resources}, 
	  meaning there may not be a human giving the robot any kind of instructions via \iaterm{external computing resources}.
	\item \textbf{Availability:} The resources must be available to use by robots of other teams as well, well before and after the competition. 
\end{enumerate}


% Local Variables:
% TeX-master: "../Rulebook"
% End:
 
