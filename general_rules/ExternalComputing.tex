%%%%%%%%%%%%%%%%%%%%%%%%%%%%%%%%%%%%%%%%%%%%%%%%%%%%%%%%%
\section{External computing}\label{rule:robot_external_computing}
Robots are allowed to use some form of external computing, for example in the form of so-called ``Cloud services'' and/or ``Internet API's'' etc. 
\begin{enumerate}
	\item \textbf{Definition:} Computing resources that are not physical part of the robot are \iterm{external computing resources}. 
	\item \textbf{Inspection:} In general, external computers are not allowed unless explained to and allowed by the \iaterm{Technical Committee}{TC}.
	  A team must announce to the TC at least 1 month in advance the external computing resources they want to use, for what purpose and how to reach the resources (e.g. specify the URL or IP address and port). Inspected software must meet the following \textbf{requirements:}
	  \begin{itemize}
	  	\item The software must be open source (BSD/GPL/etc), or
        \item Detailed information about the propietary product must be provided (e.g. vendor, patent number, licencing, pricing, etc.), as well as publishing the interface for scientific use.
	  \end{itemize}
	All relevant information must be specified in the team description paper.
	\item \textbf{Connection:} The robot may connect to \iterm{external computing resources} via a network connection, e.g. the Internet. 
	  The competition organisation cannot make any guarantees concerning availability, connectivity and performance of the connection. 
	  The robot should still be functional (albeit limited perhaps) if the \iterm{external computing resources} cannot be used for some reason.
	  This is the team's responsibility. 
	\item \textbf{Autonomy:} The robot has to maintain full autonomy when using \iterm{external computing resources}, 
	  meaning there may not be a human giving the robot any kind of instructions via \iterm{external computing resources}.
	  It is up to the team to prove to the \iaterm{Technical Committee}{TC} that there was no cheating introduced via the \iterm{external computing resources}. 
	  For example, the use of Amazon Mechanical Turk to classify and recognize objects during a competition will be considered cheating, since effectively a human will do the classification.
	  Remote control or tele-operation is also considered cheating. 
	\item \textbf{Availability:} The resources must be publicly available, for use by robots of other teams, well before and after the competition.
	\item \textbf{Recognition:} In case the resources are not developed by the team itself, the creators must be properly credited in the Team Description Paper (See \refsec{rule:website_tdp}).
	\item \textbf{Limit:} A robot is limited to use up to 5 \iterm{external computing resources}. 
\end{enumerate}

\textbf{Remark:} Teams are allowed to use their own software in the external computing devices (not only cloud services). This software must be publicly available to other teams for scientific purposes (evaluation, test, and benchmarking), as well as for TC for inspection. Although open-sourcing the software is not mandatory, this practice is advised and encouraged by the league.

% Local Variables:
% TeX-master: "../Rulebook"
% End:
 
