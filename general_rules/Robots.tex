%%%%%%%%%%%%%%%%%%%%%%%%%%%%%%%%%%%%%%%%%%%%%%%%%%%%%%%%%
\section{Robots}
\label{rule:robots}

\subsection{Autonomy \& Mobility}
Robots that participate in the RoboCup@Home league need to be \Term{autonomous}{Autonomy} and \Term{mobile}{Mobility}. Any deviations reported to the TC, may result in a penalty for the team (see \refsec{rule:extraordinary_penalties}).


\subsection{Number of robots}
\label{rule:robots_number}

\begin{enumerate}
	\item \textbf{Registration:} The maximum \term{number of robots} per team that can be registered for the competitions is \emph{two} (2).
	\item \textbf{Regular Tests:} Only one robot is allowed per test. For different tests different robots can be used.
	\item \textbf{Open Demonstrations:} In the \iterm{Open Challenge} and the \iterm{Finals} both robots can be used simultaneously.
\end{enumerate}


\subsection{Size and weight of robots}
\label{rule:robots_size}

\begin{enumerate}
	\item \textbf{Dimensions:} The dimensions of a robot should not exceed the limits of an average door, which is \SI{200}{\centi\meter} by \SI{70}{\centi\meter} in most countries.\\ 
	The TC may allow the qualification and registration of larger robots, but due to the international character of the competition it cannot be guaranteed that the robots can actually enter the arena. In case of doubt, contact the local organization. 
	\item \textbf{Weight:} There is no specific weight restriction. However, the weight of the robot and the pressure it exerts on the floor should not exceed local regulations for the construction of buildings which are used for living and/or offices in the country where the competitions is being held.
	\item \textbf{Transportation:} Team members are responsible for quickly moving the robot out of the arena.	If the robot cannot move by itself (for any reason), the team members must be able to transport the robot away with an easy and fast procedure.
\end{enumerate}



\subsection{Emergency stop button}
\label{rule:robots_emergency_button}

\begin{enumerate}
	\item \textbf{Accessibility and visibility:} Every robot has to provide an easily accessible and visible \iterm{emergency stop} button. 
	\item \textbf{Color:} It must be coloured red, and preferably be the only red button on the robot. If it is not the only red button, the TC may ask the team to tape over or remove the other red button. 
	\item \textbf{Robot behavior:} When pressing this button, the robot and all parts of it have to stop moving immediately.
	\item \textbf{Inspection:} The emergency stop button is tested during the \iterm{Robot Inspection} test (see \refsec{sec:robot_inspection}).
\end{enumerate}



\subsection{Start button}
\label{rule:start_button}

\begin{enumerate}
	\item \textbf{Requirements:} As stated in \refsec{rule:start_signal}, teams that aren't able to carry out the default start signal (opening the door) have to provide a \iterm{start button} that can be used to start tests. The team needs to announce this to the TC before every test that involves a start signal, including \iterm{Robot Inspection}.
	\item \textbf{Definition:} The start button can be any \quotes{one-button procedure} that can be easily executed by a referee.  This includes, for example, the release of the \iterm{emergency button} (\refsec{rule:robots_emergency_button}), a hardware button different from the \iterm{emergency button} (e.g., a green button), or a software button in a Graphical User Interface. 
	\item \textbf{Inspection:} It is during the the \iterm{Robot Inspection} test (see \refsec{sec:robot_inspection}) that the procedure for the start button, if needed, is announced to the TC and inspected. The start button for a robot should be the same for all the tests.
	\item \textbf{Penalty for using start button:} If a team needs to use the start button in a test where opening the door is the start signal, it may receive a penalty (see \refsec{rule:start_signal}).
\end{enumerate}



\subsection{Appearance and safety}
\label{rule:roobt_appearance}

Robots should have a nice product-like appearance, be safe to operate \& be around and should not annoy its human users. The following rules apply to all robots and are part of the \iterm{Robot Inspection} test (see \refsec{sec:robot_inspection}). 
\begin{enumerate}
	\item \textbf{Cover:} The robot's internal hardware (electronics and cables) should be covered in an appealing way. The use of (visible) duct tape is strictly prohibited.
	\item \textbf{Loose cables:} There may not be any loose cables hanging out of the robot. 
	\item \textbf{Safety:} The robot may not have sharp edges or other things that could severe people.
	\item \textbf{Annoyance:} The robot should not permanently make loud noises or use blinding lights.
	\item \textbf{Driving:} To be safe, the robots should be careful when driving in a direction it cannot sense for example. 
\end{enumerate}




\subsection{Audio output plug}\label{rule:roobt_audio_out}

\begin{enumerate}
	\item \textbf{Mandatory plug:} Either the robot or some external device connected to it \emph{must} have a \iterm{speaker output plug}. It is used to connect the robot to the sound system so that the audience and the referees can hear and follow the robot's speech output.
	\item \textbf{Inspection:} The output plug needs to be presented to the TC during the \iterm{Robot Inspection} test (see \refsec{sec:robot_inspection}).
	\item \textbf{Audio during tests:} Audio (and speech) output of the robot during a test have to be understood at least by the referees and the operators.
	\begin{compactitem}
		\item It is the responsibility of the teams to plug in the transmitter before a test, to check the sound system, and to hand over the transmitter to next team.
		\item Do not rely on the sound system! For fail-safe operation and interacting with operators make sure that the sound system is not needed, e.g., by having additional speakers directly on the robot.
\end{compactitem}
\end{enumerate}


% Local Variables:
% TeX-master: "../Rulebook"
% End:
