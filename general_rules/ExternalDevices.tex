%%%%%%%%%%%%%%%%%%%%%%%%%%%%%%%%%%%%%%%%%%%%%%%%%%%%%%%%%
\section{External devices}\label{rule:roobt_external_devices}
\begin{enumerate}
	\item \textbf{Definition:} Everything which is not part of the robot is considered an \iterm{external device}. 
	\item \textbf{Inspection:} In general, external devices are not allowed unless presented and explained to the \iaterm{Technical Committee}{TC} during the \iterm{Robot Inspection} test (see \refsec{sec:robot_inspection}).
	\item \textbf{Supervision:} In regular tests, external devices may only be used under supervision by referees and after approval by the TC. The devices have to be brought to the arena for every test, and removed quickly after the test.
	\item \textbf{Open demonstrations:} For the \iterm{Open Challenge}, \iterm{RoboZoo}, and the \iterm{Finals}, external devices are allowed, still their use needs to be announced beforehand.
	\item \textbf{Wireless devices:} All \iterm{wireless devices} including bluetooth devices, walkie-talkies, and anything else that uses an RF signal to operate need to be announced to the \iaterm{Organizing Committee}{OC}. The use of any wireless device not approved by the TC is strictly prohibited.  
	\item \textbf{Artificial landmarks:} \iterm{Artificial landmarks} and \iterm{markers} are not allowed.
	\item \textbf{Computing devices:} External computers for decentralized computations are allowed, but have to be inside the arena, i.e.,~not on its periphery.
	\item \textbf{Wireless LAN:} The use of networks other than the \iterm{arena network} (see \refsec{rule:scenario_wifi}) is strictly prohibited.
	\item \textbf{External microphones: }\iterm{External microphones}, hand microphones, and headsets are not allowed. Using an \iterm{on-board microphone} is mandatory for communication with the robot.
\end{enumerate}


% Local Variables:
% TeX-master: "../Rulebook"
% End:
