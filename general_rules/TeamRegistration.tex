%%%%%%%%%%%%%%%%%%%%%%%%%%%%%%%%%%%%%%%%%%%%%%%%%%%%%%%%%
\section{Team Registration and Qualification}


\subsection{Registration and Qualification Process}\label{rule:participation}

Each year there are four phases in the process toward participation:
\begin{enumerate}
\item \iterm{Intention of Participation} (optional)
\item \iterm{Preregistration} 
\item \iterm{Qualification} announcements
\item Final \iterm{Registration} for qualified teams
\end{enumerate}
Positions 1 and 2 will be announced by a call on the \iterm{RoboCup\char64Home mailing list}.
Preregistration requires a \iterm{team description paper}, a \Term{video}{qualification video} and a \Term{website}{Team Website}.

\subsection{Qualification Video}
As a proof of running hardware, each team has to provide a \iterm{qualification video}. 
As a minimum requirement for qualification, we consider showing the robot(s) successfully 
solving at least one test of the current or last year's rulebook.

\subsection{Team Website}

The \iterm{Team Website} has to contain photos and videos of the
robot(s), a description of the approaches, and information on
scientific achievements, relevant \iterm{publications}, team members,
and previous participation in RoboCup.

The information on the team website has to be in English and should be designed for a broader audience.

\subsection{Team Description Paper}\label{rule:website_tdp}
The \iaterm{team description paper}{TDP} should at least contain the following sections:
\begin{itemize}
\item Name of the team
\item contact information
\item website
\item team members
\item description of the hardware, including photo(s) of the robot(s)
\item description of the software
\end{itemize}

~\\\noindent Preferably, it should also contain the following:
\begin{itemize}
\item the focus of research and the contributions in the respective fields, 
\item innovative technology (if any), 
\item re-usability of the system for other research groups
\item applicability of the robot in the real world
\end{itemize}

~\\\noindent The TDP has to be in English, up to eight pages in length and formatted according to the guidelines of the RoboCup International Symposium.
It goes into detail about the technical and scientific approach.


%% %%%%%%%%%%%%%%%%%%%%%%%%%%%%%%%%%%%%%%%
\subsection{Qualification}
\label{rule:qualification}

During the \iterm{qualification process} a selection will be made by the 
%SVEN: before this was the OC not TC: 'technical committee' changed to OC.
\iaterm{Organizing Committee}{OC}
Taken into account and evaluated in this decision process are:
\begin{itemize}
\item The information on the team website and the qualification video,
\item the information in the \iterm{team description paper}, and
\item the information in the \iterm{RoboCup\char64Home Wiki} (added by the team).
\end{itemize}
(Additional) evaluation criteria are: 
\begin{itemize}
\item the performance in previous competitions, 
\item the relevant scientific contributions and publications, and
\item the contributions to the RoboCup@Home league.
\end{itemize}
For getting considered in the evaluation, be sure to insert your team's name 
when adding information to the \iterm{RoboCup\char64Home Wiki}.    

