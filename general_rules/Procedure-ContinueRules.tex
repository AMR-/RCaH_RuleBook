\subsection{CONTINUE rule: Because the show must go on}
\label{rule:continue}
Having problems with certain particular ability should not disqualify a robot from showing up what it can do best, for demonstrating robots' abilities is important in RoboCup@Home.

To prevent this situation, when a robot is unable to perform a
speech recognition
or manipulation
task, it may request human assistance and continue with the test; however, no points are scored for solving the involved task. The Referee of the test determines the applicability of the CONTINUE rule and whether points are scored.

\subsubsection{Bypassing Automatic Speech Recognition}
\label{rule:asrcontinue}
Giving commands to the robot is essential in many tests. To foster natural human-robot interaction, speech has been chosen as the primary way to command a robot; but Automatic Speech Recognition (ASR) is not infallible.

Because active robots are preferred over robots that are passive due to failing ASR, teams are allowed to provide means to bypass ASR via an Alternative method for HRI (see \refsec{rule:asralternative}).

\paragraph{Procedure}
Automatic Speech Recognition is preferred and any command given to the robot will given by voice first.
\begin{enumerate}
	\item \textbf{Default Operator:} The command for the robot is spoken out loud and clear by the human operator. This grants 100\% of the available points for understanding the command. The \iterm{default operator} may repeat the command up to three times.

	\item \textbf{Custom Operator:} When the robot renders unable to understand the default operator, the team leader can choose a \iterm{custom operator} can give the command \emph{exactly as instructed by the referee}. Unless stated otherwise, only 75\% of the points are granted. A \iterm{custom operator} may repeat the command up to three times.

	\item \textbf{Alternative Input Method:} When the robot renders unable to understand the command given by a \iterm{custom operator}, it is allowed to use any alternative method or interface (see \refsec{rule:asralternative}) previously approved by the TC during the Robot Inspection (see \refsec{sec:robot_inspection}). No points are scored this way.
\end{enumerate}


\paragraph{Alternative input methods for HRI}
\label{rule:asralternative}
Alternative methods for HRI offer a way for a robot to start or complete a task. Any reasonable method may be used, with the following criteria:
\begin{itemize}
	\item \textbf{Intuitive to use and self-explanatory:} a manual should not be needed. Teams are not allowed to explain how to interface with the robot. %you immediately know how to use it after a quick glance

	\item \textbf{Effortless use:} Must be as easy to use as uttering a command. %is as easy to use as it is uttering a command

	\item \textbf{Is smart and preemptive:} The interface adapts to the user input, displaying only the options that make sense or that the robot can actually perform.

	\item Exploits the best of the device being used (eg. touch screen, display area, speakers, etc.)
\end{itemize}

Preferably, the alternative HRI must be also adapted to the user. Consider localization (with English as the default), but also potential users of service robots at their home. For example: elderly people and people with physical disabilities.

\textbf{\textsc{Award:}} The best alternative is awarded the Best Human-Robot Interface award (\refsec{award:hri}).

% Below are some suggested alternatives for ASR:
% \begin{itemize}
%   \item A touch-sensitive designed interface.
%   \item Other types of natural interaction such as gestures.
%   \item ...
% \end{itemize}

% What a good custom ASR alternative is loosely defined on purpose to foster and allow creativity. 

\subsubsection{Bypassing Manipulation}
\label{rule:mancontinue}
Manipulating objects is a desirable feature for a domestic robot that also is required in many tests. However, due to design constrains, or even to malfunctioning a robot could fail a test when unable to grasp an object. To prevent this situation, robots that are aware of their limited manipulation capabilities can request human assistance during manipulation.

\paragraph{Procedure}
Autonomous manipulation is desired, however, it is preferred for a robot to request human assistance over damaging itself, the furniture or the objects.
\begin{enumerate}
	\item \textbf{Attempt autonomous manipulation:} Optional. The robot may try (but is not required) to manipulate the object by itself. Robots can request assistance after a failed attempt.

	\item \textbf{State intention:} When the robot renders unable to manipulate an object, it must request for human assistance, clearly stating the nature of the assistance, such as opening a door, uncapping a bottle, grasping an object, etc.

	\item \textbf{Closed-loop HRI:} When asking for assistance, the robot must be aware of the human's actions, like indicating which door to open, which object to take, or by guiding the human during the operation (eg. telling when to stop doing something).\\

	When grasping or moving objects, the robot needs to clearly specify (and, when possible, point out) the properties of the object to take or move (eg. specifying relative size, colour, shape, type, etc.), confirming that the human assistant has taken the correct object and, when required, also the exact placing location and transportation procedure.

	\item \textbf{Thank for the help:} Robot must be polite and thank the human once the interaction has finished (e.g. once the door has been opened).
\end{enumerate}

\subparagraph{Remark:} When using the CONTINUE rule to bypass autonomous manipulation, it is not possible to also use the CONTINUE rule to bypass Automated Speech Recognition.

% Local Variables:
% TeX-master: "../Rulebook"
% End:
