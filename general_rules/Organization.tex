%%%%%%%%%%%%%%%%%%%%%%%%%%%%%%%%%%%%%%%%%%%%%%%%%%%%%%%%%
\section{Organization of the competition}
\label{sec:procedure_during_competition}

\subsection{Stage system}\label{rule:stages}

The competition features a \iterm{stage system}. It is organized in two stages each consisting of a number of specific tests. It ends with the \iterm{Finals}.

\begin{enumerate}
	\item \textbf{Stage~I:} The first days of the competition will be called \iterm{Stage~I}. 
	All qualified teams can participate in \iterm{Stage~I}. Stage~I comprehends a set of \iterm{Ability Tests}, an \iterm{Integration Test}, and an audience demonstration called \iterm{Following \& Guiding}. 
	Those \iterm{Proficency Tests} (\iterm{Ability Tests}, and \iterm{Integration Test}) are performed multiple times (See \refsec{rule:score_system}). 

% MAURICIO @2017: With the inclusion of SPL only 6 teams per league advance to the second stage (sub-leagues are intended to have 13 teams).
	\item \textbf{Stage~II:} The best \emph{50\% of teams with full integrated capabilities}\footnote{If the total number of teams is less than 12, up to 6 teams may advance to Stage~II} (after Stage~I) advance to \iterm{Stage~II}. Here, more complex abilities or combinations of abilities are tested. In order to advance to Stage~II a team must successfully solve 3 out of \iterm{Proficency Tests} in Stage~I. \\
	The \iterm{Open Challenge} is the open demonstration in Stage~II.
% MAURICIO @2017: With the inclusion of SPL, we want the finals with no more than 6 teams. At this point, the best two of each sub-league advance to the finals.
	\item \textbf{Final demonstration:} The best \emph{two teams} of each sub-league, the ones with the highest score after Stage~II, advance to the final round. The final round features only a single open demonstration.
\end{enumerate}

% MAURICIO: No technical challenge since 2015
% In addition, a Technical Challenge (see \refsec{sec:TechnicalChallenge}) is carried out between Stage~II and the Final Demonstration, and its schedule is outside the scope of the Stage system.
In case of having no considerable score deviation between a team advancing to the next stage and a team dropping out, the TC may announce additional teams advancing to the next stage.


% MAURICIO @2017: With the inclusion of SPL, we make mandatory participating in at least one Stage II test to advance to the finals
\subsection{Number of tests}\label{rule:number_of_tests}
None of the tests is mandatory, except for the \iterm{Robot Inspection} test (see \refsec{sec:robot_inspection}). However, in order to participate in the finals, a team must have participated in at least one test of the Stage~II.


%%%%%%%%%%%%%%%%%%%%%%%%%%%%%%%%%%%%%%%%%%%%%%%%%%%%%%%%%
\subsection{Schedule}
\label{rule:schedule}

\begin{enumerate}
	\item \textbf{Tests:} The \iaterm{Organizing Committee}{OC} provides schedules for all tests and teams. 
	\item \textbf{Participation is default:} Teams have to indicate to the \iaterm{Organizing Committee}{OC} in which tests they are \emph{not} going to participate. Without such indication, they are automatically added to all test schedules and may receive a penalty when not attending (see \refsec{rule:not_attending}).
	\item \textbf{Slots:} The tests will be held in \iterm{test slots} of approximately two hours.
	\item \textbf{Preparation:} The \iaterm{Organizing Committee}{OC} provides schedules for all teams to organize the access to the arena between test slots. In these \iterm{preparation slots} the teams may conduct calibration procedures, remap the arena if necessary, or conduct test runs.
	Preparation slots are inserted whenever possible, but may not be available before all test slots. 
	\item \textbf{Arena access:} One hour before a test slot, only the teams participating in that slot are allowed in the arena.
This rule only applies when not having organized \iterm{preparation slots}.   
\end{enumerate}


%%%%%%%%%%%%%%%%%%%%%%%%%%%%%%%%%%%%%%%%%%%%%%%%%%%%%%%%%
%\subsection{Score system}\label{rule:score_system}
%
%\begin{enumerate}
%  \item \textbf{Stage~I:} The maximum total score per test in Stage~I is \scoring{2000 points}.
%  \item \textbf{Stage~II:} The maximum total score per test in Stage~II is \scoring{2600 points}.
%  \item \textbf{Special tests:} Tests may specify a maximum total score deviating from the general maximum total scores.  
%  \item \textbf{Minimum score:} The minimum total score per test in Stage~I and Stage~II is \scoring{0 points}. 
%  That is, if the total score for a test is below zero, the team does not receive any points.
%  \item \textbf{Penalties:} An exception to the \emph{minimum score} rule are penalties. 
%  Both penalties for not attending (see \refsec{rule:not_attending}) and extraordinary 
%  penalties (see \refsec{rule:extraordinary_penalties}) can cause a total negative score. 
%  \item \textbf{Partial scores:} All tests---except for the open demonstrations---are rewarded on a partial scoring basis. 
%  \begin{enumerate}
%  \item Tests are split into designated parts.
%  \item Each part is assigned a certain number of points.
%  \item A team that successfully passes a designated part of the test receives points for that part.
%  \item In case of partial success, referees (and TC members) may decide to only award a percentage instead of the full partial score.  
%  \item The total score for a test is the sum of partial scores.
%  \item Partial scores can be negative (e.g.~to penalize failures etc.).
%  \end{enumerate}
%\end{enumerate}

% MAURICIO: Explained Score System
\subsection{Score system}
\label{rule:score_system}

\begin{enumerate}
	\item \textbf{Stage~I:} The maximum total score (excluding special penalties and bonuses) in \iterm{Stage~I} is \scoring{1150 points}.
	% \item \textbf{Stage~I:} The maximum total score (excluding special penalties and bonuses) in \iterm{Stage~I} is \scoring{1050 points}.
	\begin{enumerate}
		\item \textbf{\iterm{Proficency Tests}:} Each proficiency test is attempted three times. The maximum total score is calculated as the average of the best two attempts for that test.
		% \item \textbf{Poster Session:} The Poster Session score is counted in Stage~I and may grant up to \scoring{50 points}.
	\end{enumerate}

	\item \textbf{Stage~II:} Test in \iterm{Stage~II} are rewarded on a task-solved scoring basis.
	\begin{enumerate}
		\item Each test but the \iterm{Open Challenge} has a main task. The base score for solving the main task is \scoring{250 points}.
		\item The maximum score for \iterm{Open Challenge} is \scoring{250 points}.
		\item Optionals and subtasks add bonus points to the main task score.
	\end{enumerate}

	\item \textbf{\iterm{Finals}:} Final score is normalized and special evaluation is used

	\item \textbf{Special tests:} Tests may specify a maximum total score deviating from the general maximum total scores.

	\item \textbf{Minimum score:} The minimum total score per test in \iterm{Stage~I} and \iterm{Stage~II} is \scoring{0 points}. That is, if the total score for a test is below zero, the team does not receive any points.

	\item \textbf{Penalties:} An exception to the \emph{minimum score} rule are penalties. Both penalties for not attending (see \refsec{rule:not_attending}) and extraordinary penalties (see \refsec{rule:extraordinary_penalties}) can cause a total negative score. 

	\item \textbf{Partial scores:} All tests---except for the open demonstrations---are rewarded on a partial scoring basis. 
	\begin{enumerate}
		\item Tests are split into designated parts.
		\item Each part is assigned a certain number of points.
		\item A team that successfully passes a designated part of the test receives points for that part.
		\item In case of partial success, referees (and TC members) may decide to only award a percentage instead of the full partial score.  
		\item The total score for a test is the sum of partial scores.
		\item Partial scores can be negative (e.g.~to penalize failures etc.).
	\end{enumerate}
\end{enumerate}


%%%%%%%%%%%%%%%%%%%%%%%%%%%%%%%%%%%%%%%%%%%%%%%%%%%%%%%%%
% MAURICIO: On 2015, Open Challenge moved to Stage 2. There is no Demo Challenge
\subsection{Open Demonstrations}
\label{sec:open-demonstrations}
\begin{enumerate}
	\item \textbf{Stage~II:} The \iterm{Open Challenge} is the open demonstration in \iterm{Stage~II}.
	\begin{enumerate}
		\item To participate in the \iterm{Open Challenge}, a team needs to participate in at least one regular \iterm{Stage~II} test.
		\item Teams can demonstrate freely chosen abilities. 
		\item The performance is evaluated by a jury consisting of the \iaterm{Technical Committee}{TC}.
		\item The \iterm{Open Challenge} is described in \refsec{sec:test_open_challenge}.
	\end{enumerate}

	% \item \textbf{Stage~II:} The \iterm{Demo Challenge} is the open demonstration in Stage~II.
	% \begin{enumerate}
	% \item To participate in the Demo Challenge, a team needs to participate in at least one regular Stage~II test.
	% 	\item The scope (and topic) of the Demo Challenge are defined by the TC on a yearly basis.
	% 	\item Teams can demonstrate freely chosen abilities, but according to the scope. 
	% 	\item The performance is evaluated by the \iaterm{Technical Committee}{TC}.
	% 	\item The Demo Challenge is described in \refsec{sec:test_demo_challenge}.
	% \end{enumerate}
	
	\item \textbf{\iterm{Finals}:} The competition ends with a final demonstration.
	\begin{enumerate}
		\item The concept of the final demonstration is the same as that of the \iterm{Open Challenge}, but the performance evaluation is different. 
		\item The are two juries---an \emph{external} consisting of three or more people not from the RoboCup @Home league, and an \emph{internal} formed by the \iaterm{Executive Committee}{EC}. Both juries have different sets of evaluation criteria.
		\item Members of the external jury are selected by the \iaterm{Executive Committee}{EC} on site. 
		\item The demonstration in the \iterm{Finals} does not have to be different from the one shown in the \iterm{Open Challenge}. It does not have to be the same either.
	\end{enumerate}
\end{enumerate}


% Local Variables:
% TeX-master: "../Rulebook"
% End:
