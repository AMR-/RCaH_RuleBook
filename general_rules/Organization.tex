%%%%%%%%%%%%%%%%%%%%%%%%%%%%%%%%%%%%%%%%%%%%%%%%%%%%%%%%%
\section{Divisions}
RoboCup@Home features three divisions:
\begin{itemize}
 \item \textbf{RoboCup@Home Social Standard Platform} which uses an Aldebaran Pepper social robot.
 \item \textbf{RoboCup@Home Domestic Standard Platform} which uses a Toyota HSR robot.
 \item \textbf{RoboCup@Home Open Platform} which uses custom made robots. 
\end{itemize}
Robots of all divisons must meet the requirements in \refsec{rule:robots}. 

These three divisions all use the same rules, although with minor modifications where absolutely necessary. 

\section{Organization of the competition}
\label{sec:procedure_during_competition}

The competition consists of a number of challenges, which are developed to be repeatable, quick to set up and easily score-able by the referees. 
Each challenge features a (number of) application relevant task(s). 
A challenge is judged on the overall success and, with a less significant amount of points, on the subtasks for achieving a task. 

None of the tests is mandatory, except for the \iterm{Robot Inspection} test (see \refsec{sec:robot_inspection}).

%%%%%%%%%%%%%%%%%%%%%%%%%%%%%%%%%%%%%%%%%%%%%%%%%%%%%%%%%
\subsection{Schedule}
\label{rule:schedule}

\begin{enumerate}
	\item \textbf{Tests:} The \iaterm{Organizing Committee}{OC} provides schedules for all tests and teams. 
	\item \textbf{Participation is default:} Teams have to indicate to the \iaterm{Organizing Committee}{OC} in which tests they are \emph{not} going to participate. Without such indication, they are automatically added to all test schedules and may receive a penalty when not attending (see \refsec{rule:not_attending}).
	\item \textbf{Slots:} The tests will be held in \iterm{test slots} of approximately two hours.
	\item \textbf{Preparation:} The \iaterm{Organizing Committee}{OC} provides schedules for all teams to organize the access to the arena between test slots. In these \iterm{preparation slots} the teams may conduct calibration procedures, remap the arena if necessary, or conduct test runs.
	Preparation slots are inserted whenever possible, but may not be available before all test slots. 
	\item \textbf{Arena access:} One hour before a test slot, only the teams participating in that slot are allowed in the arena.
This rule only applies when not having organized \iterm{preparation slots}.   
\end{enumerate}

% MAURICIO: Explained Score System
\subsection{Score system}
\label{rule:score_system}

\begin{enumerate}
	\begin{enumerate}
		\item \textbf{\iterm{Challenges}:} Each challenge is attempted three times. The maximum total score is calculated as the average of the best two attempts for that challenge.
	\end{enumerate}

	\item Each test but the \iterm{Open Challenge} has a main task. The base score for solving the main task is dependent on it relative difficulty. 
	\item The maximum score for \iterm{Open Challenge} is \scoring{250 points}.
	\item Optionals and subtasks add bonus points to the main task score.

	\item \textbf{\iterm{Finals}:} Final score is normalized and special evaluation is used

	\item \textbf{Special tests:} Tests may specify a maximum total score deviating from the general maximum total scores.

	\item \textbf{Minimum score:} The minimum total score per challenge is \scoring{0 points}. That is, if the total score for a test is below zero, the team does not receive any points.

	\item \textbf{Penalties:} An exception to the \emph{minimum score} rule are penalties. Both penalties for not attending (see \refsec{rule:not_attending}) and extraordinary penalties (see \refsec{rule:extraordinary_penalties}) can cause a total negative score. 

	\item \textbf{Partial scores:} All tests---except for the open demonstrations---are rewarded on a partial scoring basis. 
	\begin{enumerate}
		\item Tests are split into designated parts.
		\item Each part is assigned a certain number of points.
		\item A team that successfully passes a designated part of the test receives points for that part.
		\item In case of partial success, referees (and TC members) may decide to only award a percentage instead of the full partial score.  
		\item The total score for a test is the sum of partial scores.
		\item Partial scores can be negative (e.g.~to penalize failures etc.).
	\end{enumerate}
\end{enumerate}


%%%%%%%%%%%%%%%%%%%%%%%%%%%%%%%%%%%%%%%%%%%%%%%%%%%%%%%%%
\subsection{Open Demonstrations}
\label{sec:open-demonstrations}
\begin{enumerate}
	\item The \iterm{Open Challenge} is the open demonstration during the regular challenges.
	\begin{enumerate}
		\item To participate in the \iterm{Open Challenge}, a team needs to participate in at least one regular challenge.
		\item Teams can demonstrate freely chosen abilities. 
		\item The performance is evaluated by a jury consisting of the \iaterm{Technical Committee}{TC}.
		\item The \iterm{Open Challenge} is described in \refsec{sec:test_open_challenge}.
	\end{enumerate}
	
	\item \textbf{\iterm{Finals}:} The competition ends with a final demonstration.
	\begin{enumerate}
		\item The concept of the final demonstration is the same as that of the \iterm{Open Challenge}, but the performance evaluation is different. 
		\item The are two juries---an \emph{external} consisting of three or more people not from the RoboCup @Home league, and an \emph{internal} formed by the \iaterm{Executive Committee}{EC}. Both juries have different sets of evaluation criteria.
		\item Members of the external jury are selected by the \iaterm{Executive Committee}{EC} on site. 
		\item The demonstration in the \iterm{Finals} does not have to be different from the one shown in the \iterm{Open Challenge}. It does not have to be the same either.
	\end{enumerate}
\end{enumerate}


% Local Variables:
% TeX-master: "../Rulebook"
% End:
