\subsection{Standard Platform Leagues}
RoboCup@Home features two Standard Platform Leagues adhering to the rules listed above.

\subsubsection{Modifications}
\label{rule:spl-mods}
The idea of having standardized platform is to allow teams to compete in equality of conditions by eliminating all hardware-dependent variables. Therefore, both Standard Platform Leagues are considered as \emph{closed hardware design}, meaning that and modifications and alterations to the robots are strictly forbidden; including, but not limited to attaching, connecting, plugging, gluing, and taping components into and onto the robot, as well as modifying or altering the robot structure. Voiding this rule leads to immediate disqualification from the competition and penalty for the team (see \refsec{rule:extraordinary_penalties}).

All robots competing in a Standard Platform League will be inspected by TC during the \iterm{Robot Inspection} test (see \refsec{sec:robot_inspection}), who will verify that the robot is in proper state for the competition, presenting no alterations and in neat condition. In addition, EC and TC members may request re-inspection of a SPL robot at any time during the competition.

\textbf{Clothing, coloring, and stickers:} Robots competing in a Standard Platform League are allowed to \quotes{wear} clothes, as well as have stickers (e.g., a sticker exhibiting the logo of an sponsor). Painting the robot with another color or design is also allowed. However, teams must keep in mind that no artificial markers are allowed when personalizing the appearance or a robot. This includes, but is not limited to bar codes, QR codes, OpenCV markers, fluorescent and phosphorescent colors, and reflective stickers. Finally, is important to remark that teams should contact first the robot's vendor and review the lease contract to verify they are authorized to alter the robot's appearance. 

\subsubsection{Domestic Standard Platform League}
The characteristics of the Toyota Human Support Robot are detailed below. 

\begin{itemize}
    \item Aimed at human support tasks, elderly care et cetera
    \item Omni-directional base, maximum speed 0.8km/h
    \item 1 arm with multifunctional gripper through a vacuum pad. The wrist is equipped with a force-torque sensor. Capable of lifting 1.2kg. 
    \item RGB-D, stereo cameras and wide-angle camera
    \item Display mounted in head, separate tablet interface
    \item Access to cloud-based services
    \item Equipped with a microphone array
    \item Gravity compensated arm
    \item Height-adjusting torso
\end{itemize}

\subsubsection{Social Standard Platform League}
The characteristics of the Softbank Robotics/Aldebaran Pepper are detailed below. 

\begin{itemize}
    \item Aimed at social interaction, public environments, explainable artificial intelligence
    \item Omni-directional base, maximum speed 3km/h
    \item 2 arms mostly intended for social gesturing. 
    \item 3D and 2 HD cameras
    \item Equipped with a built-in tablet
    \item Access to cloud-based services
    \item Equipped with a 4-microphone array in the head
    \item Emotion recognition by voice and images
    \item Emotion engine to adapt it's attitude
\end{itemize}