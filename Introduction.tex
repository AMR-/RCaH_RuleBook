%% %%%%%%%%%%%%%%%%%%%%%%%%%%%%%%%%%%%%%%%%%%%%%%%%%%%%%%%%%%%%%%%%%%%%%%%%%%%
%%
%%    author(s): RoboCupAtHome Technical Committee(s)
%%  description: Introduction
%%
%% %%%%%%%%%%%%%%%%%%%%%%%%%%%%%%%%%%%%%%%%%%%%%%%%%%%%%%%%%%%%%%%%%%%%%%%%%%%
\chapter{Introduction}
\label{chap:introduction}


\section{RoboCup}
\iterm{RoboCup} is an international joint project to promote AI, robotics, and related fields. It is an attempt to foster AI and intelligent robotics research by providing standard problems where a wide range of technologies can be integrated and examined. More information can be found at http://www.robocup.org/.

\section{RoboCup@Home}
The \iterm{RoboCup@Home} league aims to develop service and assistive robot technology with high relevance for future personal domestic applications. It is the largest international annual competition for autonomous service robots and is part of the RoboCup initiative. A set of benchmark tests is used to evaluate the robots abilities and performance in a realistic non-standardized home environment setting. Focus lies on the following domains but is not limited to: Human-Robot-Interaction and Cooperation, Navigation and Mapping in dynamic environments, Computer Vision and Object Recognition under natural light conditions, Object Manipulation, Adaptive Behaviors, Behavior Integration, Ambient Intelligence, Standardization and System Integration. It is collocated with the RoboCup symposium.

\section{Organization}

\subsection{Executive Committee --- ec@robocupathome.org}
\label{sec:ec}
The \iaterm{Executive Committee}{EC} consists of members of the board of trustees, and representatives of each activity area. Members representing the @Home league:
\begin{itemize}
\item Dirk Holz (University of Bonn, Germany)
\item Maja Rudinac ( Delft University of Technology, Netherlands)
\item Sven Wachsmuth (Bielefeld University, Germany)
\end{itemize}

\subsection{Technical Committee --- tc@robocupathome.org}
\label{sec:tc}
The \iaterm{Technical Committee}{TC} is responsible for the rules of each league. Members of the RoboCup@Home Technical Committee for \YEAR:
\begin{itemize}
\item Kai Chen (University of Science and Technology of China, China)
\item Caleb Rascon (Universidad Nacional Aut{\'o}noma de M{\'e}xico, Mexico)
\item Loy Van Beek (Eindhoven University of Technology, The Netherlands)
\item Mauricio Matamoros  (Delft University of Technology, The Netherlands)
\item Josemar Rodrigues de Souza (Bahia State University, Brazil)
\item Hideoki Nagano (Tokyo City University, Japan)
\item Loreto Martinez Luz Sanchez (University of Chile, Chile)
\end{itemize}
The Technical Committee also includes the members of the Executive Committee.

\subsection{Organizing Committee --- oc@robocupathome.org}
\label{sec:oc}
The \iaterm{Organizing Committee}{OC} is responsible for the organization of the competition. Members of the RoboCup@Home Organizing Committee for \YEAR:

\begin{itemize}
\item (chair) Sebastian Meyer zu Borgsen (Bielefeld University, Germany)
\item Krit Chaiso (Kasetsart University, Thailand)
\item Fagner Pimentel (Bahia State University, Brazil)
\item Raphael Memmesheimer (University of Koblenz, Germany)
\item He Chauncey (Beijing Information Science \& Technology University, China)
\item Local chairs:
  \begin{itemize}
    \item Paul G. Ploeger (Bonn-Rhein-Sieg University of Applied Sciences, Germany) 
    \item Dirk Holz (University of Bonn, Germany)
  \end{itemize}
\end{itemize}

\section{Infrastructure}
\label{sec:infrastructure}
\subsection{RoboCup@Home Mailinglist}
The official \iterm{RoboCup@Home mailing list} can be reached at
\begin{center}
\texttt{robocup-athome@lists.robocup.org}
\end{center}
You can register to the email list at:
\begin{center}
http://lists.robocup.org/listinfo.cgi/robocup-athome-robocup.org
\end{center}

\subsection{RoboCup@Home Web Page}
The official \iterm{RoboCup@Home website} that also hosts this RuleBook can be found at \\
\begin{center}
http://www.robocupathome.org/
\end{center}

% \subsection{RoboCup@Home Wiki}
% \label{sec:at_home_wiki}
% The official \iterm{RoboCup@Home Wiki} is meant to be a central place to collect information on all topics related to the RoboCup@Home league. It was set up to simplify and unify the exchange of relevant information. This includes but is certainly not limited to hardware, software, media, data, and alike. The \textit{wiki} can be reached at \\
% \begin{center}
% http://robocup.rwth-aachen.de/athomewiki.
% \end{center}
% To contribute, i.e.~to add/edit/change things you need to create an account and log in.

\section{Competition}
The competition consists of a set of task-driven \iterm{Challenges} and the \iterm{Finals}. 
Each challenge involves tasks that are being held in a daily life environment and are relevant in a daily life environment. 
The competition ends with the \emph{Finals} where only the five highest ranked teams compete to become the winner.

\section{Awards}
The RoboCup@Home league features the following \iterm{awards}.

\subsection{Winner of the competition}
There will be a 1st, 2nd, and 3rd place award.

\subsection{Innovation award}
To honour outstanding technical and scientific achievements as well as applicable solutions in the @Home league, a special \iterm{innovation award} may be given to one of the participating teams. Special attention is being paid to making usable robot components and technology available to the @Home community.

The \iaterm{Executive Committee}{EC} members from the RoboCup@Home league nominate a set of candidates for the award. The \iaterm{Technical Committee}{TC} elects the winner. A TC member whose team is among the nominees is not allowed to vote.

There is no innovation award in case no outstanding innovation and no nominees, respectively.

% \subsection{Winner of the Technical Challenge}
% In parallel to the regular competition, the RoboCup@Home league features a \iterm{technical challenge}. The winner of the technical challenge is given a special \iterm{award for winning the technical challenge}.
%
% As with the innovation award, the award for winning the technical challenge is not given in case no team shows a \emph{sufficient performance}. The decision which team wins the technical challenge, and if the award is given at all, is conducted by the \iaterm{Technical Committee}{TC}.

\subsection{Skill Certificates}
  The @Home league features certificates for the robots best at a the skills below:
  \begin{itemize}
   \item Navigation
   \item Manipulation
   \item Speech Recognition
   \item Person Recognition
  \end{itemize}
  
  A team is given the certificate if it scored at least 75\% of the attainable points for that skill.
  This is counted over all challenges, so e.g. if the robot scores manipulation points during the navigation test to open the door, that will count for the Manipulation-certificate.
  The certificate will only be handed out if the team is \emph{not} the overall winner of the competition. 

% Local Variables:
% TeX-master: "Rulebook"
% End:
