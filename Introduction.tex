%% %%%%%%%%%%%%%%%%%%%%%%%%%%%%%%%%%%%%%%%%%%%%%%%%%%%%%%%%%%%%%%%%%%%%%%%%%%%
%%
%%    author(s): RoboCupAtHome Technical Committee(s)
%%  description: Introduction
%%
%% %%%%%%%%%%%%%%%%%%%%%%%%%%%%%%%%%%%%%%%%%%%%%%%%%%%%%%%%%%%%%%%%%%%%%%%%%%%
\chapter{Introduction}
\label{chap:introduction}


\section{RoboCup}
\iterm{RoboCup} is an international joint project to promote AI, robotics, and related fields. It is an attempt to foster AI and intelligent robotics research by providing standard problems where a wide range of technologies can be integrated and examined. More information can be found at http://www.robocup.org/.

\section{RoboCup@Home}
The \iterm{RoboCup@Home} league aims to develop service and assistive robot technology with high relevance for future personal domestic applications. It is the largest international annual competition for autonomous service robots and is part of the RoboCup initiative. A set of benchmark tests is used to evaluate the robots abilities and performance in a realistic non-standardized home environment setting. Focus lies on the following domains but is not limited to: Human-Robot-Interaction and Cooperation, Navigation and Mapping in dynamic environments, Computer Vision and Object Recognition under natural light conditions, Object Manipulation, Adaptive Behaviors, Behavior Integration, Ambient Intelligence, Standardization and System Integration. It is collocated with the RoboCup symposium.

\section{Organization}

\subsection{Executive Committee --- ec@robocupathome.org}
\label{sec:ec}
The \iaterm{Executive Committee}{EC} consists of members of the board of trustees, and representatives of each activity area. Members representing the @Home league:
\begin{itemize}
\item Dirk Holz (University of Bonn, Germany)
\item Javier Ruiz del Solar (Department of Electric Engineering, Universidad de Chile, Chile)
\item Maja Rudinac ( Delft University of Technology, Netherlands)
\item Sven Wachsmuth (Bielefeld University, Germany)
\end{itemize}

\subsection{Technical Committee --- tc@robocupathome.org}
\label{sec:tc}
The \iaterm{Technical Committee}{TC} is responsible for the rules of each league. Members of the RoboCup@Home Technical Committee for 2016:
\begin{itemize}
\item Kai Chen (University of Science and Technology of China, China)
\item Caleb Rascon (Universidad Nacional Aut�noma de M{\'e}xico, Mexico)
\item Loy Van Beek (Eindhoven University of Technology, The Netherlands)
\item Mauricio Matamoros  (Delft University of Technology, The Netherlands)
\end{itemize}
The Technical Committee also includes the members of the Executive Committee.

\subsection{Organizing Committee --- oc@robocupathome.org}
\label{sec:oc}
The \iaterm{Organizing Committee}{OC} is responsible for the organization of the competition. Members of the RoboCup@Home Organizing Committee for 2016:

\begin{itemize}
\item Chair: Maja Rudinac (Delft University of Technology, The Netherlands)
\item Local chair: Yinfeng Chen (University of Science and Technology of China, China)
\item Farshid Abdollahi (Qazvin Islamic Azad University, Iran)
\item Sammy Pfeifer (Pal Robotics, Spain)
\item Sebastian Meyer zu Borgsen (Bielefeld University, Germany)
\item Viktor Seib (Universitaet Koblenz-Landau, Germany)
\end{itemize}

\section{Infrastructure}
\label{sec:infrastructure}
\subsection{RoboCup@Home Mailinglist}
The official \iterm{RoboCup@Home mailing list} can be reached at
\begin{center}
\texttt{robocup-athome@lists.robocup.org}
\end{center}
You can register to the email list at:
\begin{center}
http://lists.robocup.org/listinfo.cgi/robocup-athome-robocup.org
\end{center}

\subsection{RoboCup@Home Web Page}
The official \iterm{RoboCup@Home website} that also hosts this RuleBook can be found at \\
\begin{center}
http://www.robocupathome.org/
\end{center}

% \subsection{RoboCup@Home Wiki}
% \label{sec:at_home_wiki}
% The official \iterm{RoboCup@Home Wiki} is meant to be a central place to collect information on all topics related to the RoboCup@Home league. It was set up to simplify and unify the exchange of relevant information. This includes but is certainly not limited to hardware, software, media, data, and alike. The \textit{wiki} can be reached at \\
% \begin{center}
% http://robocup.rwth-aachen.de/athomewiki.
% \end{center}
% To contribute, i.e.~to add/edit/change things you need to create an account and log in.

\section{Competition}
The competition consists of 2 \emph{Stages} and the \iterm{Finals}. Each stage consists of a series of \iterm{Tests} that are being held in a daily life environment. The best teams from \iterm{Stage~I} advance to \iterm{Stage~II} which consists of more difficult tests. The competition ends with the \emph{Finals} where only the five highest ranked teams compete to become the winner.

\section{Awards}
The RoboCup@Home league features the following \iterm{awards}.

\subsection{Winner of the competition}
There will be a 1st, 2nd, and 3rd place award.

\subsection{Innovation award}
To honour outstanding technical and scientific achievements as well as applicable solutions in the @Home league, a special \iterm{innovation award} may be given to one of the participating teams. Special attention is being paid to making usable robot components and technology available to the @Home community.

The \iaterm{Executive Committee}{EC} members from the RoboCup@Home league nominate a set of candidates for the award. The \iaterm{Technical Committee}{TC} elects the winner. A TC member whose team is among the nominees is not allowed to vote.

There is no innovation award in case no outstanding innovation and no nominees, respectively.

% \subsection{Winner of the Technical Challenge}
% In parallel to the regular competition, the RoboCup@Home league features a \iterm{technical challenge}. The winner of the technical challenge is given a special \iterm{award for winning the technical challenge}.
%
% As with the innovation award, the award for winning the technical challenge is not given in case no team shows a \emph{sufficient performance}. The decision which team wins the technical challenge, and if the award is given at all, is conducted by the \iaterm{Technical Committee}{TC}.

\subsection{Winner of the RoboZoo}
The winner of the \iterm{RoboZoo} in the category of performance is given a special \iterm{award for wining the RoboZoo}. The decision of which team wins have the robot which performs best is made by an open audience during the RoboZoo test, however, as with the innovation award, the award for winning the RoboZoo is not given in case the team with the highest score didn't show \emph{sufficient performance} according by the the \iaterm{Technical Committee}{TC} criteria.

\subsection{Best Test Score Certificate}
The team with the highest score in each test in Stage 1 and Stage 2 is given a \iterm{Best Test Score Certificate}. To qualify for this award, the obtained score must be at least 70\% of the maximum score for that test (see \refsec{sec:best_score_certificate}).

\subsection{Best looking robot}
The winner of the \iterm{RoboZoo} in the category of appearance challenge is given a special \iterm{award for best looking robot}. The decision of which team wins have the best looking robot is made by an open audience during the RoboZoo test.

\subsection{Most functional robot}
The winner of the \iterm{RoboZoo} in the category of functionality challenge is given a special \iterm{award for most functional robot}. The decision of which team wins have the most functional robot is made by an open audience during the RoboZoo test.


% Local Variables:
% TeX-master: "Rulebook"
% End:
